\chapter{Results}
\label{cha:results}

In this chapter the results generated with the open source tool are presented.
This chapter is divided into two sections.
Section \ref{sec:results_verification} presents results from previous papers recreated using the tool.
Section \ref{sec:results_new} presents results from the newer family of models introduced in \cite{Ferrer2018a}

\section{Verification of previous results}
\label{sec:results_verification}

In this section a replication of previous results is presented.
This allows to verify the model and its implementation against previous experiments.

However, in some cases there was limited information in the original papers about the parameters used in the experiments.
This concerned details regarding the initial graphs and the specifics of the optimization process.
In other cases, there were errors in previous papers.
This was due to undetected programming errors.
In both cases, determining the correct parameter becomes a matter of trial and error.

The $\phi$ parameter was not considered in either of the papers replicated in this section, and so it is set to 0.

Section \ref{sec:results_verification_first} replicates the results from \cite{Ferrer2005a}, which correspond to the \firstmodel{} from this thesis.
Section \ref{sec:results_verification_second} replicates the results from \cite{Ferrer2003a}, corresponding to the \secondmodel{}.

\subsection{Results from the \firstmodel{} (2005)}
\label{sec:results_verification_first}

\begin{figure}
  \addjankysubfigure{a)}{\includegraphics[width=\textwidth,height=.35\textheight,keepaspectratio]{fig2_2005_old}}
  \addjankysubfigure{b)}{\includegraphics[width=\textwidth,height=.35\textheight,keepaspectratio]{figure2_article_EPJ_B_2005}}
  \caption{
    The mutual information $I(S,R)$ is in the $y$ axis and the $\lambda$ parameter on the $x$ axis.
    Graphs of different sizes are shown, $n=m=100$, $n=m=200$, $n=m=300$, $n=m=400$.
    Averages over 30 realizations.
    $\phi=0$, the initial graph is \randgraph{n}{m}{\frac{1}{n}}, each iteration of the optimization algorithm performs 2 mutations on the graph, the weak stop condition is used to stop the optimization process, unlinked objects are allowed.\\
    Subfigure a corresponds with Figure 2 from \cite{Ferrer2005a}.\\
    Subfigure b is the recreation of that same figure.
  }
  \label{fig:fig2_2005}
\end{figure}

\begin{figure}
  \addjankysubfigure{a)}{\includegraphics[width=\textwidth,height=.3\textheight,keepaspectratio]{fig3_2005_old}}
  \addjankysubfigure{b)}{\includegraphics[width=\textwidth,height=.3\textheight,keepaspectratio]{figure3_article_EPJ_B_2005}}
  \caption{
    $P(i)$, the probability of the $i$th most frequent signal, obtained from minimum energy configurations for systems of sizes $n=m=100$ (A), $n=m=200$ (B), $n=m=300$ (C), $n=m=400$ (D).
    Four series are shown in each plot, $\lambda=0.49$ (circles), $\lambda=\lambda^*$ (squares) and $\lambda=0.5$ (diamonds) and the ideal curve for $\alpha^*$.
    Averages over 30 realizations.
    When $\lambda=\lambda^*$, $\alpha^* = 1.54$ for $n=m=100$, $\alpha^* = 1.51$ for $n=m=200$, $\alpha^* = 1.5$ for $n=m=300$ and $\alpha^* = 1.49$ for $n=m=400$.
    It was chosen that $\lambda^* = 0.4986$ for $n=m=100$, $\lambda^* = 0.4987$ for $n=m=200$, $\lambda^* = 0.4987$ for $n=m=300$, and $\lambda^* = 0.4986$ for $n=m=400$.
    Averages over 30 realizations.
    $\phi=0$, the initial graph is \randgraph{n}{m}{\frac{1}{n}}, each iteration of the optimization algorithm performs 2 mutations on the graph, the weak stop condition is used to stop the optimization process, unlinked objects are allowed.\\
    Subfigure a corresponds with Figure 3 from \cite{Ferrer2005a}.\\
    Subfigure b is the recreation of that same figure.
  }
  \label{fig:fig3_2005}
\end{figure}

Figure \ref{fig:fig2_2005} shows both the original Figure 2 from \cite{Ferrer2005a} and a recreation of this figure using the new tool.
Figure \ref{fig:fig3_2005} shows both the original Figure 3 from \cite{Ferrer2005a} and a recreation of this figure, also generated using the new tool.

The initial graph was not specified in \cite{Ferrer2005a} and a \randgraph{n}{m}{\frac{1}{n}} ($n=m$) graph is used in the replication.
The paper also does not specify how to stop the optimization process and so the weak stop condition (Section \ref{sec:methods_optimization}) is used.

Subfigures a and b from Figure \ref{fig:fig2_2005} are nearly identical.
In Figure \ref{fig:fig3_2005}, subfigures a and b are also qualitatively very similar.
Although some of the points can be seen to be not quite in the same place.

\subsection{Results from the \secondmodel{} (2003)}
\label{sec:results_verification_second}

\begin{figure}
  \addjankysubfigure{a)}{\includegraphics[width=\textwidth]{fig2_2003_old}}
  \addjankysubfigure{b)}{\includegraphics[width=\textwidth]{figure2_article_PNAS_2003_binom1.75}}
  \caption{
    In A, $I(S,R)$ is the mutual information obtained for values of $\lambda$ between 0 and 1.
    In B, $L$ is the lexicon size obtained for values of $\lambda$ between 0 and 1.
    An abrupt change is seen for $\lambda \approx 0.41$ in both A and B.
    Averages over 30 replicas.
    $n=m=150$, $\phi=0$, unlinked objects are not allowed, $\pi$ follows a uniform distribution.\\
    Subfigure a corresponds with Figure 2 from \cite{Ferrer2003a}\\
    Subfigure b is the recreation of that same figure using the open source tool created for this thesis.
  }
  \label{fig:fig2_2003}
\end{figure}

\begin{figure}
  \addjankysubfigure{a)}{\includegraphics[width=\textwidth]{fig3_2003_old}}
  \addjankysubfigure{b)}{\includegraphics[width=\textwidth]{figure3_article_PNAS_2003_binom1.75}}
  \caption{
    Normalized signal frequency $P(k)$ versus rank $k$.
    The dashed lines show the distribution obtained when from a \randgraph{n}{m}{e^*} graph where $e^*$ is the number of connections in these optimal configurations.
    In both cases the distribution of B is still consistent with human language, $\alpha=1$.
    Averages over 30 replicas, $n=m=150$, the initial graph is \randgraph{n}{m}{\frac{10}{n}}, the number of mutations on each iteration follows a binomial distribution with average 1.75, the optimization stops after $2nm$ mutations that do not improve the cost function.\\
    Above: Figure 3 from \cite{Ferrer2003a}\\
    Below: Recreation of that same figure using the open source tool created for this thesis.
  }
  \label{fig:fig3_2003}
\end{figure}

Figure \ref{fig:fig2_2003} shows both the original Figure 2 from \cite{Ferrer2003a} and the recreation of this figure using the new tool.
Figure \ref{fig:fig3_2003} shows the original Figure 3 from \cite{Ferrer2003a} and a recreation using the new tool.

The initial graph was specified as a \randgraph{n}{m}{\rho} but the value of $\rho$ was not given in the paper, a \randgraph{n}{m}{\frac{10}{n}} graph was used (that is, a value of $\rho=\frac{10}{n}$).
It is specified that the number of mutations done on each iteration of the optimization algorithm follows a binomial distribution.
However it was later found that, due to a bug in the generator of binomial numbers, the number of mutations must have been lower.
Several values were tried for the average number of binomial mutations, with 1.75 giving the results most similar to the previous model's.

Qualitatively, it seems that the two subfigures in Figure \ref{fig:fig2_2003} are very similar.
Some of the points in A do not form exactly the same shape, however, and it seems that not as many points appear inside the phase transition.
As for Figure \ref{fig:fig3_2003}, subfigures A and C are quite similar (although the dashed line appears to overlap the data in subfigure b more than it did in a).
Subfigure B is not as similar.
However, the slope of the curve is also 1.0 as indicated by the solid line.

\section{Results for current models}
\label{sec:results_new}

Here aspects of these models that had not been considered before are investigated, such as untested linguistic laws or regions of their parameter space.
All results presented here come from the minimization of the cost function (Equation \eqref{eq:definition-Omega}).
Firstly, and as a way to for an intuition of the kinds of graphs that these models generate, Section \ref{sec:results_new_graph} presents the optimal graphs for different initial conditions for both models.
Section \ref{sec:results_new_other} shows the rest of the obtained results.
This section focuses first on plots of the values obtained for the information theoretical measures of the optimal graphs for various value of $\lambda$ in the cost function.
As a reminder, $\lambda$ controls the weight of either the entropy or the mutual information in the cost function.
Statistical measures for selected values of $\lambda$ are then shown.
These measures show how various linguistic laws are recovered in the results.
Several kinds of initial graphs are shown throughout this section.
\begin{redenv}
  For each of the four models (\firstmodel{} with $\phi=0$ and $\phi\neq 0$, \secondmodel with $\phi=0$ and $\phi\neq 0$) studied, several initial conditions are shown.
  The aim is to investigate how these particular initial conditions can influence the development of the model.
\end{redenv}

\begin{redenv}
  \begin{itemize}
  \item Random: The initial graph is a \randgraph{n}{m}{\frac{3}{nm}} random graph.
  \item Single link: The initial graph contains a single link between a word and a meaning. \redtxt{This initial condition is also a global minimum of the cost function for $\lambda \leq \frac{1}{2}$ (See Section \ref{TODO})}
  \item One-to-one: The initial graph is a bijection connecting words and meanings one to one,
    \redtxt{which is a minimum of the cost function when $\lambda \geq \frac{1}{2}$ (Section \ref{TODO})}
  \item Complete: The initial graph is a complete bipartite graph where every word connects to every meaning and every meaning to every word. \redtxt{This is a maximum of the cost function.}
  \end{itemize}
\end{redenv}

All results in this section perform two mutations on each iteration of the optimization algorithm, which stops with the \emph{weak} stop condition.
See Section \ref{sec:methods_optimization} for more information on the optimization process.

\subsection{Graph visualization}
\label{sec:results_new_graph}

Here various graphs are plotted.
They correspond to graphs of size $n=m=60$ and $\phi=1$.
\begin{redenv}
  The graph is initialized with some initial condition and then the optimization process is executed.
  The presented graph is the result of this optimization process.
\end{redenv}
Unlinked meanings are also allowed in all graphs.
It is interesting to see the different shapes the graph takes for different values of $\lambda$.

For the first model, four graphs are shown.

\begin{redenv}
  \begin{itemize}
  \item Random initial graph, Figure \ref{fig:graphVisualization_firstModel_phi1_nm60_static_randomBipartite_allowUnlinked}.
    For lower values of $\lambda$ a single word dominates, becoming linked to almost every meaning.
    For greater values of $\lambda$ more words appear linked to other meanings.
    This persists for $\lambda \leq 0.5$ is reached.
    For $\lambda=1$ the graphs are composed of long chains of connected words and meanings.
  \item Single link initial graph, Figure \ref{fig:graphVisualization_firstModel_phi1_nm60_static_singleLink_allowUnlinked}.
    A single link is maintained for $\lambda \leq 0.5$.
    After that point, the graph evolves into an almost one-to-one configuration of links between words and meanings (some words or meanings may remain unlinked).
  \item One-to-one initial graph, Figure \ref{fig:graphVisualization_firstModel_phi1_nm60_static_oneToOne_allowUnlinked}.
    The graph evolves to a configuration where a single word dominates for $\lambda < 0.5$.
    However, after $\lambda=0.5$ is reached, the graphs remain in a one-to-one configuration of links between words and meanings, which in this case becomes difficult to see as every word and every meaning are linked.
  \item Complete initial graph, Figure \ref{fig:graphVisualization_firstModel_phi1_nm60_static_complete_allowUnlinked}.
    Behaves similarly to the random initial graph, with most meanings converging on one word while all other words and meanings remain unconnected for $\lambda \leq 0.5$.
    Also for $\lambda=1$ it can be seen that the graphs are composed of long chains of connected words and meanings.
  \end{itemize}
\end{redenv}

For the second model, the \emph{a priori} probability $\pi$ follows a uniform distribution.
Four additional graphs are shown.

\begin{redenv}
  \begin{itemize}
  \item Random initial graph, Figure \ref{fig:graphVisualization_uniform_phi1_nm60_static_randomBipartite_allowUnlinked}.
    For $\lambda < 0.5$, a single word dominates and becomes connected to most meanings.
    For $\lambda \geq 0.5$ less meanings are attached to the word.
    When $\lambda$ has reached 1, it seems that a single meaning begins to dominate words, with most other words linked in a one-to-one configuration with the rest of meanings.
  \item Single link initial graph, Figure \ref{fig:graphVisualization_uniform_phi1_nm60_static_singleLink_allowUnlinked}.
    Similarly to the \firstmodel{}, the single link does not evolve and is maintained for $\lambda \leq 1$.
    When $\lambda$ has reached 1 we see the same one-to-one configuration with one dominating meaning as in the random initial condition.
  \item One-to-one initial graph, Figure \ref{fig:graphVisualization_uniform_phi1_nm60_static_oneToOne_allowUnlinked}.
    For $\lambda < 0.5$ a single word dominates, while for $\lambda \geq 0.5$ the initial condition fails to evolve into another configuration.
  \item Complete initial graph, Figure \ref{fig:graphVisualization_uniform_phi1_nm60_static_complete_allowUnlinked}.
    Presents almost the same behavior as the random initial graph, with a single dominating word at the start which weakens and, by the time $\lambda$ has reached 1, a single meaning seems to dominate instead.
  \end{itemize}
\end{redenv}

\begin{figure}
  \centering
  \includegraphics[height=\textheight,width=\textwidth,keepaspectratio]{graphVisualization_firstModel_phi1_nm60_static_randomBipartite_allowUnlinked}
  \caption{
    Samples of optimized graphs for various values of $\lambda$.
    The graphs correspond to $\lambda=0$ (a, b), $\lambda=0.49$ (c,d), $\lambda=0.5$ (e,f) and $\lambda=1$ (g,h).
    The white circles represent words while the black ones represent meanings.
    These graphs follow the \firstmodel{} with $\phi=1$.
    The initial condition is a random graph.
  }
  \label{fig:graphVisualization_firstModel_phi1_nm60_static_randomBipartite_allowUnlinked}
\end{figure}

\begin{figure}
  \centering
  \includegraphics[height=\textheight,width=\textwidth,keepaspectratio]{graphVisualization_firstModel_phi1_nm60_static_singleLink_allowUnlinked}
  \caption{Same as Figure \ref{fig:graphVisualization_firstModel_phi1_nm60_static_randomBipartite_allowUnlinked} but the initial condition is a single link.}
  \label{fig:graphVisualization_firstModel_phi1_nm60_static_singleLink_allowUnlinked}
\end{figure}

\begin{figure}
  \centering
  \includegraphics[height=\textheight,width=\textwidth,keepaspectratio]{graphVisualization_firstModel_phi1_nm60_static_oneToOne_allowUnlinked}
  \caption{Same as Figure \ref{fig:graphVisualization_firstModel_phi1_nm60_static_randomBipartite_allowUnlinked} but the initial condition is a one to one configuration.}
  \label{fig:graphVisualization_firstModel_phi1_nm60_static_oneToOne_allowUnlinked}
\end{figure}

\begin{figure}
  \centering
  \includegraphics[height=\textheight,width=\textwidth,keepaspectratio]{graphVisualization_firstModel_phi1_nm60_static_complete_allowUnlinked}
  \caption{Same as Figure \ref{fig:graphVisualization_firstModel_phi1_nm60_static_randomBipartite_allowUnlinked} but the initial condition is a complete bipartite graph.}
  \label{fig:graphVisualization_firstModel_phi1_nm60_static_complete_allowUnlinked}
\end{figure}

\begin{figure}
  \centering
  \includegraphics[height=\textheight,width=\textwidth,keepaspectratio]{graphVisualization_uniform_phi1_nm60_static_randomBipartite_allowUnlinked}
  \caption{Same as Figure \ref{fig:graphVisualization_firstModel_phi1_nm60_static_randomBipartite_allowUnlinked} but it follows the \secondmodel{}.}
  \label{fig:graphVisualization_uniform_phi1_nm60_static_randomBipartite_allowUnlinked}
\end{figure}

\begin{figure}
  \centering
  \includegraphics[height=\textheight,width=\textwidth,keepaspectratio]{graphVisualization_uniform_phi1_nm60_static_singleLink_allowUnlinked}
  \caption{Same as Figure \ref{fig:graphVisualization_firstModel_phi1_nm60_static_randomBipartite_allowUnlinked} but it follows the \secondmodel{} and the initial condition is a single link.}
  \label{fig:graphVisualization_uniform_phi1_nm60_static_singleLink_allowUnlinked}
\end{figure}

\begin{figure}
  \centering
  \includegraphics[height=\textheight,width=\textwidth,keepaspectratio]{graphVisualization_uniform_phi1_nm60_static_oneToOne_allowUnlinked}
  \caption{Same as Figure \ref{fig:graphVisualization_firstModel_phi1_nm60_static_randomBipartite_allowUnlinked} but it follows the \secondmodel{} and the initial condition is a one to one configuration.}
  \label{fig:graphVisualization_uniform_phi1_nm60_static_oneToOne_allowUnlinked}
\end{figure}

\begin{figure}
  \centering
  \includegraphics[height=\textheight,width=\textwidth,keepaspectratio]{graphVisualization_uniform_phi1_nm60_static_complete_allowUnlinked}
  \caption{Same as Figure \ref{fig:graphVisualization_firstModel_phi1_nm60_static_randomBipartite_allowUnlinked} but it follows the \secondmodel{} and the initial condition is a complete bipartite graph.}
  \label{fig:graphVisualization_uniform_phi1_nm60_static_complete_allowUnlinked}
\end{figure}


\subsection{Information theoretic and statistical measures}
\label{sec:results_new_other}

This section shows the data obtained from executing simulations with various sets of parameters.
All simulations are run with a graph size of $n=m=400$.
Every experiment in this section is the result of averaging at least 20 realizations.
Most experiments consist of 100 realizations.
Some had to be reduced to 20 realizations due to the total running time and are marked as such.
Results for $\phi=0$ and $\phi=1$ are shown.
Three different initial conditions are tested: random bipartite graph, single link and one-to-one.
For the \secondmodel{} only results for $\pi$ following a uniform probability distribution are shown.
The chosen $\lambda$ value for which curves are fitted is the one that's qualitatively closer to a power law.

For the \firstmodel{} with $\phi=0$, Figures \ref{fig:informationTheoretic_firstModel_phi0_nm400_dynamic_randomBipartite_allowUnlinked}, \ref{fig:informationTheoretic_firstModel_phi0_nm400_dynamic_singleLink_allowUnlinked} and \ref{fig:informationTheoretic_firstModel_phi0_nm400_dynamic_oneToOne_allowUnlinked} show the information theoretic measures of the optimal graph for values of $\lambda$ ranging from 0 to 1.
They correspond to several distinct initial conditions.
\redtxt{TODO: descriure figures, phase transition at 0.5, disturbances just before 0.5}

\begin{figure}
  \centering
  \includegraphics[height=0.7\textheight]{informationTheoretic_firstModel_phi0_nm400_dynamic_randomBipartite_allowUnlinked}
  \caption{
    Information theoretic and graph measures for the optimized graph as a function of $\lambda$.
    The graph follows the equations of the \firstmodel{} with $\phi=0$ and the initial condition of the optimization process is a random bipartite graph \randgraph{n}{m}{\frac{3}{nm}}. Disconnected meanings are allowed.
    On the $x$ axis of each subfigure is the parameter $\lambda$ of the optimization process, ranging from 0 to 1.
    On the $y$ of the subfigures is: the mutual information between words and meanings (a), the joint entropy between words and meanings (b), the entropy of words (c), the entropy of meanings (d), the conditional entropy of words given the meanings (e), the conditional entropy of the meanings given the words (f), the number of referentially useless words (g) and the proportion of the largest connected component of the graph (h).
    Averages over 100 realizations.
}
  \label{fig:informationTheoretic_firstModel_phi0_nm400_dynamic_randomBipartite_allowUnlinked}
\end{figure}

\begin{figure}
  \centering
  \includegraphics[height=0.7\textheight]{informationTheoretic_firstModel_phi0_nm400_dynamic_singleLink_allowUnlinked}
  \caption{Same information as in Figure \ref{fig:informationTheoretic_firstModel_phi0_nm400_dynamic_randomBipartite_allowUnlinked} but for a single link as the initial condition.}
  \label{fig:informationTheoretic_firstModel_phi0_nm400_dynamic_singleLink_allowUnlinked}
\end{figure}

\begin{figure}
  \centering
  \includegraphics[height=0.7\textheight]{informationTheoretic_firstModel_phi0_nm400_dynamic_oneToOne_allowUnlinked}
  \caption{Same information as in Figure \ref{fig:informationTheoretic_firstModel_phi0_nm400_dynamic_randomBipartite_allowUnlinked} but for one to one connections between signals and meanings as the initial condition.}
  \label{fig:informationTheoretic_firstModel_phi0_nm400_dynamic_oneToOne_allowUnlinked}
\end{figure}

Figures \ref{fig:insideLambda_firstModel_phi0_nm400_dynamic_randomBipartite_allowUnlinked} and \ref{fig:insideLambda_firstModel_phi0_nm400_dynamic_oneToOne_allowUnlinked} show statistical measures of selected values of $\lambda$, with Figures \ref{fig:fitting_insideLambda_firstModel_phi0_nm400_dynamic_randomBipartite_allowUnlinked} and \ref{fig:fitting_insideLambda_firstModel_phi0_nm400_dynamic_oneToOne_allowUnlinked} showing the fitting of the curve to a power law for a single selected value of $\lambda$.
It can be seen that a power law (linear in log-log scale) appears in the plots.
Tables \ref{tab:fitting_insideLambda_firstModel_phi0_nm400_dynamic_randomBipartite_allowUnlinked} and \ref{tab:fitting_insideLambda_firstModel_phi0_nm400_dynamic_oneToOne_allowUnlinked} show the values of the regression exponent and factor.
\redtxt{TODO: descriure figures, single link not plotted}

\begin{figure}
  \centering
  \includegraphics[height=0.7\textheight]{insideLambda_firstModel_phi0_nm400_dynamic_randomBipartite_allowUnlinked}
  \caption{
    Statistical measures of the optimized graph for selected values of $\lambda$.
    The graph follows the equations of the \firstmodel{} with $\phi=0$ and the initial condition of the optimization process is a random bipartite graph \randgraph{n}{m}{\frac{3}{nm}}. Disconnected meanings are allowed.
    The green line corresponds to $\lambda=0.49$, the blue line to $\lambda=0.5$ and the red line to $\lambda=\lambda^*$.
    The value of $\lambda^*$ is chosen qualitatively and in this case $\lambda^*=0.4989$.
    Figure (a) shows the probability (or frequency) of a word as a function of its probability rank.
    Figure (b) shows the cumulative proportion of the frequency of words.
    Figure (c) shows the number of meanings of a word as a function of its probability rank.
    Figure (d) shows the number of meanings of a word as a function of its probability.
    Figure (e) shows age of a word as a function of its probability rank.
    Figure (f) shows the number of meanings of a word as a function of its degree rank.
    Figure (g) shows the cumulative proportion of the number of meanings of words.
    All figures are in a log-log scale.
    Averages over 100 realizations.
  }
  \label{fig:insideLambda_firstModel_phi0_nm400_dynamic_randomBipartite_allowUnlinked}
\end{figure}

\begin{figure}
  \centering
  \includegraphics[height=0.7\textheight]{insideLambda_firstModel_phi0_nm400_dynamic_oneToOne_allowUnlinked}
  \caption{Same information as in Figure \ref{fig:insideLambda_firstModel_phi0_nm400_dynamic_randomBipartite_allowUnlinked} but the initial condition is one to one connections between words and meanings. $\lambda^* = 0.4989$}
  \label{fig:insideLambda_firstModel_phi0_nm400_dynamic_oneToOne_allowUnlinked}
\end{figure}

\begin{figure}
  \centering
  \includegraphics[height=0.65\textheight]{fitting_insideLambda_firstModel_phi0_nm400_dynamic_randomBipartite_allowUnlinked}
  \caption{
    Statistical measures of the optimized graph for a selected value of $\lambda=\lambda^*$.
    The graph follows the equations of the \firstmodel{} with $\phi=0$ and the initial condition of the optimization process is a random bipartite graph \randgraph{n}{m}{\frac{3}{nm}}. Disconnected meanings are allowed.
    $\lambda^*=0.4989$.
    The black line indicates the values predicted by the Theil-Sen liniear regression.
    The red line indicates the actual values in the graph.
    Figure (a) shows the probability (or frequency) of a word as a function of its probability rank.
    Figure (b) shows the cumulative proportion of the frequency of words.
    Figure (c) shows the number of meanings of a word as a function of its probability rank.
    Figure (d) shows the number of meanings of a word as a function of its probability.
    Figure (f) shows the number of meanings of a word as a function of its degree rank.
    Figure (g) shows the cumulative proportion of the number of meanings of words.
    Figure (e), which should show the age of a word as a function of its probability rank, is omitted as it never followed a power law.
    All figures are in a log-log scale.
    Table \ref{tab:fitting_insideLambda_firstModel_phi0_nm400_dynamic_randomBipartite_allowUnlinked} shows the values of the exponent and the factor of the fitted power law.
  }
  \label{fig:fitting_insideLambda_firstModel_phi0_nm400_dynamic_randomBipartite_allowUnlinked}
\end{figure}

\begin{figure}
  \centering
  \includegraphics[width=\textwidth]{fitting_insideLambda_firstModel_phi0_nm400_dynamic_oneToOne_allowUnlinked}
  \caption{Same information as in Figure \ref{fig:fitting_insideLambda_firstModel_phi0_nm400_dynamic_randomBipartite_allowUnlinked} but the initial condition is one to one connections between words and meanings. $\lambda^*=0.
4989$.
Table \ref{tab:fitting_insideLambda_firstModel_phi0_nm400_dynamic_oneToOne_allowUnlinked} shows the values of the exponent and the factor of the fitted power law.}
  \label{fig:fitting_insideLambda_firstModel_phi0_nm400_dynamic_oneToOne_allowUnlinked}
\end{figure}

% latex table generated in R 4.0.4 by xtable 1.8-4 package
% Wed Mar 17 14:20:58 2021
\begin{table}
\centering
\begin{tabular}{lrr}
  \hline
plot & $\alpha$ & $k$ \\ 
  \hline
a & 1.5131590 & 0.4415009 \\ 
  b & 0.6406267 & 0.0016491 \\ 
  c & 1.4972982 & 170.7165580 \\ 
  d & -0.9992804 & 403.8279052 \\ 
  f & 1.4972982 & 170.7165580 \\ 
  g & 0.6725912 & 0.0750000 \\ 
   \hline
\end{tabular}
\caption{
  Table showing the exponent ($\alpha$) and the factor ($k$) of the power laws fitted in Figure \ref{fig:fitting_insideLambda_firstModel_phi0_nm400_dynamic_randomBipartite_allowUnlinked} for each of the subfigures. The power law follows the formula $y = kx^{-\alpha}$.
} 
\label{tab:fitting_insideLambda_firstModel_phi0_nm400_dynamic_randomBipartite_allowUnlinked}
\end{table}


\begin{table}
  \centering
  \begin{adjustbox}{max width=\textwidth}
    \begin{tabular}{llSS}
      \toprule
      plot & law & $a$ & $k$ \\ 
      \midrule
      a & word frequency & 4.0992847 & 2.3970141 \\ 
      b & \redtxt{word frequency (cumulative)} & 0.2466051 & 0.0030639 \\ 
      c & meaning distribution & 3.9365300 & 577.6844467 \\ 
      d & meaning frequency & -0.9673668 & 271.7315918 \\ 
      f & \redtxt{meaning distribution} & 3.9365300 & 577.6844467 \\ 
      g & \redtxt{meaning distribution (cumulative)} & 0.2731698 & 0.0125000 \\ 
      \bottomrule
    \end{tabular}
  \end{adjustbox}
  \caption{\redtxt{en vermell: cal mostrar?} Table showing the exponent and factor of the power laws fitted in Figure \ref{fig:fitting_insideLambda_firstModel_phi0_nm400_dynamic_oneToOne_allowUnlinked}.} 
  \label{tab:fitting_insideLambda_firstModel_phi0_nm400_dynamic_oneToOne_allowUnlinked}
\end{table}

%%% Local Variables:
%%% mode: latex
%%% TeX-master: "../tfm"
%%% End:



The next set of figures correspond also to the \firstmodel{} but for $\phi=1$, Figures  \ref{fig:informationTheoretic_firstModel_phi1_nm400_dynamic_randomBipartite_allowUnlinked},  \ref{fig:informationTheoretic_firstModel_phi1_nm400_dynamic_singleLink_allowUnlinked} and  \ref{fig:informationTheoretic_firstModel_phi1_nm400_dynamic_oneToOne_allowUnlinked} show the information theoretic measures of the optimal graph for values of $\lambda$ ranging from 0 to 1.
They correspond to several distinct initial conditions
\redtxt{TODO: descriure figures, phase transition at 0.5, disturbances just before 0.5}.

\begin{figure}
  \centering
  \includegraphics[height=0.7\textheight]{informationTheoretic_firstModel_phi1_nm400_dynamic_randomBipartite_allowUnlinked}
  \caption{Same information as in Figure \ref{fig:informationTheoretic_firstModel_phi0_nm400_dynamic_randomBipartite_allowUnlinked} but with $\phi=1$.}
  \label{fig:informationTheoretic_firstModel_phi1_nm400_dynamic_randomBipartite_allowUnlinked}
\end{figure}

\begin{figure}
  \centering
  \includegraphics[height=0.7\textheight]{informationTheoretic_firstModel_phi1_nm400_dynamic_singleLink_allowUnlinked}
  \caption{Same information as in Figure \ref{fig:informationTheoretic_firstModel_phi0_nm400_dynamic_randomBipartite_allowUnlinked} but with $\phi=1$ and for a single link as the initial condition.}
  \label{fig:informationTheoretic_firstModel_phi1_nm400_dynamic_singleLink_allowUnlinked}
\end{figure}

\begin{figure}
  \centering
  \includegraphics[height=0.7\textheight]{informationTheoretic_firstModel_phi1_nm400_dynamic_oneToOne_allowUnlinked}
  \caption{Same information as in Figure \ref{fig:informationTheoretic_firstModel_phi0_nm400_dynamic_randomBipartite_allowUnlinked} but with $\phi=1$ and for one to one connections between signals and meanings as the initial condition.}
  \label{fig:informationTheoretic_firstModel_phi1_nm400_dynamic_oneToOne_allowUnlinked}
\end{figure}

Figures \ref{fig:insideLambda_firstModel_phi1_nm400_dynamic_randomBipartite_allowUnlinked} and \ref{fig:insideLambda_firstModel_phi1_nm400_dynamic_oneToOne_allowUnlinked} show statistical measures of selected values of $\lambda$, with Figures \ref{fig:fitting_insideLambda_firstModel_phi1_nm400_dynamic_randomBipartite_allowUnlinked} and \ref{fig:fitting_insideLambda_firstModel_phi1_nm400_dynamic_oneToOne_allowUnlinked} showing the fitting of the curve to a power law for a single selected value of $\lambda$.
In this case it's hard to say that the curves follow a power law.
Nevertheless, Tables \ref{tab:fitting_insideLambda_firstModel_phi1_nm400_dynamic_randomBipartite_allowUnlinked} and \ref{tab:fitting_insideLambda_firstModel_phi1_nm400_dynamic_oneToOne_allowUnlinked} show the values of the regression exponent and factor of the attempt at fitting them as power laws.
\redtxt{TODO: no initial condition}

\begin{figure}
  \centering
  \includegraphics[height=0.7\textheight]{insideLambda_firstModel_phi1_nm400_dynamic_randomBipartite_allowUnlinked}
  \caption{Same information as in Figure \ref{fig:insideLambda_firstModel_phi0_nm400_dynamic_randomBipartite_allowUnlinked} but $\phi=1$. $\lambda^* = 0.4993$}
  \label{fig:insideLambda_firstModel_phi1_nm400_dynamic_randomBipartite_allowUnlinked}
\end{figure}

\begin{figure}
  \centering
  \includegraphics[height=0.7\textheight]{insideLambda_firstModel_phi1_nm400_dynamic_oneToOne_allowUnlinked}
  \caption{Same information as in Figure \ref{fig:insideLambda_firstModel_phi0_nm400_dynamic_randomBipartite_allowUnlinked} but $\phi=1$ and the initial condition is one to one connections between words and meanings. $\lambda^* = 0.4986$}
  \label{fig:insideLambda_firstModel_phi1_nm400_dynamic_oneToOne_allowUnlinked}
\end{figure}

\begin{figure}
  \centering
  \includegraphics[width=\textwidth]{fitting_insideLambda_firstModel_phi1_nm400_dynamic_randomBipartite_allowUnlinked}
  \caption{Same information as in Figure \ref{fig:fitting_insideLambda_firstModel_phi0_nm400_dynamic_randomBipartite_allowUnlinked} but $\phi=1$. $\lambda^*=0.4986$.
Table \ref{tab:fitting_insideLambda_firstModel_phi1_nm400_dynamic_randomBipartite_allowUnlinked} shows the values of the exponent and the factor of the fitted power law.}
  \label{fig:fitting_insideLambda_firstModel_phi1_nm400_dynamic_randomBipartite_allowUnlinked}
\end{figure}

\begin{figure}
  \centering
  \includegraphics[width=\textwidth]{fitting_insideLambda_firstModel_phi1_nm400_dynamic_oneToOne_allowUnlinked}
  \caption{Same information as in Figure \ref{fig:fitting_insideLambda_firstModel_phi0_nm400_dynamic_randomBipartite_allowUnlinked} but $\phi=1$ and the initial condition is one to one connections between words and meanings. $\lambda^*=0.4993$.
Table \ref{tab:fitting_insideLambda_firstModel_phi1_nm400_dynamic_oneToOne_allowUnlinked} shows the values of the exponent and the factor of the fitted power law.}
  \label{fig:fitting_insideLambda_firstModel_phi1_nm400_dynamic_oneToOne_allowUnlinked}
\end{figure}

% latex table generated in R 4.0.4 by xtable 1.8-4 package
% Wed Mar 17 14:32:51 2021
\begin{table}[ht]
\centering
\begin{tabular}{lrr}
  \hline
plot & alpha & k \\ 
  \hline
a & -0.2871434 & 0.0000468 \\ 
  b & -2.5493723 & 0.0000000 \\ 
  c & -0.3072861 & 2.8668125 \\ 
  d & 1.0876280 & 149669.1453764 \\ 
  f & -0.3072861 & 2.8668125 \\ 
  g & -0.5944739 & 0.0650000 \\ 
   \hline
\end{tabular}
\caption{caption} 
\label{tables/fitting_insideLambda_firstModel_phi1_nm400_dynamic_randomBipartite_allowUnlinked}
\end{table}


\begin{table}
  \centering
  \begin{adjustbox}{max width=\textwidth}
    \begin{tabular}{llSS[scientific-notation=true]}
      \toprule
      plot & law & $a$ & $k$ \\ 
      \midrule
      a & word frequency ($\alpha$) & 0.2433438 & 0.0000524 \\ 
      %b & \redtxt{word frequency (cumulative)} & 2.2902299 & 0.0000000 \\ 
      b & meaning distribution ($\gamma$) & 0.3100715 & 3.2784604 \\ 
      c & meaning frequency ($\delta$) & -1.2166546 & 501643.7856344 \\ 
      %f & \redtxt{meaning distribution} & 0.3100715 & 3.2784604 \\ 
      %g & \redtxt{meaning distribution (cumulative)} & 0.4063538 & 0.0225000 \\ 
      \bottomrule
    \end{tabular}
  \end{adjustbox}
  \caption{Table showing the exponent and factor of the power laws fitted for the \firstmodel{} with $\phi=1$ and a one to one configuration as the initial condition (Figure \ref{fig:fitting_insideLambda_firstModel_phi1_nm400_dynamic_oneToOne_allowUnlinked})
    In this table, $\alpha \approx 0.25$, $\delta \approx 1.2$ and $\gamma \approx 0.3$.
    The relationship between these values (Equation \eqref{eq:relation-exponents}) does not hold.
    The exponents are not exactly the ones expected, although $\gamma$ is close.
  }
  \label{tab:fitting_insideLambda_firstModel_phi1_nm400_dynamic_oneToOne_allowUnlinked}
\end{table}

%%% Local Variables:
%%% mode: latex
%%% TeX-master: "../tfm"
%%% End:


The next sets of figures correspond to the \secondmodel{} with $\phi=0$.
Figures \ref{fig:informationTheoretic_uniform_phi0_nm400_dynamic_randomBipartite_allowUnlinked},  \ref{fig:informationTheoretic_uniform_phi0_nm400_dynamic_singleLink_allowUnlinked} and  \ref{fig:informationTheoretic_uniform_phi0_nm400_dynamic_oneToOne_allowUnlinked} show the information theoretic measures of the optimal graph for values of $\lambda$ ranging from 0 to 1.
\redtxt{TODO: descriure figures, phase transition at 0.5, disturbances just before 0.5}
Appendix \ref{sec:app_figures_second-model} shows the same figures but with disconnected meanings disallowed.

\begin{figure}
  \centering
  \includegraphics[height=0.7\textheight]{informationTheoretic_uniform_phi0_nm400_dynamic_randomBipartite_allowUnlinked}
  \caption{Same information as in Figure \ref{fig:informationTheoretic_firstModel_phi0_nm400_dynamic_randomBipartite_allowUnlinked} but the graph uses the equations of the \secondmodel{} with $\pi$ following a uniform distribution.}
  \label{fig:informationTheoretic_uniform_phi0_nm400_dynamic_randomBipartite_allowUnlinked}
\end{figure}

\begin{figure}
  \centering
  \includegraphics[height=0.7\textheight]{informationTheoretic_uniform_phi0_nm400_dynamic_singleLink_allowUnlinked}
  \caption{Same information as in Figure \ref{fig:informationTheoretic_firstModel_phi0_nm400_dynamic_randomBipartite_allowUnlinked} but the graph uses the equations of the \secondmodel{} with $\pi$ following a uniform distribution and the initial condition is a single link}
  \label{fig:informationTheoretic_uniform_phi0_nm400_dynamic_singleLink_allowUnlinked}
\end{figure}

\begin{figure}
  \centering
  \includegraphics[height=0.7\textheight]{informationTheoretic_uniform_phi0_nm400_dynamic_oneToOne_allowUnlinked}
  \caption{Same information as in Figure \ref{fig:informationTheoretic_firstModel_phi0_nm400_dynamic_randomBipartite_allowUnlinked} but the graph uses the equations of the \secondmodel{} with $\pi$ following a uniform distribution and the initial condition is one to one connections between signals and meanings.}
  \label{fig:informationTheoretic_uniform_phi0_nm400_dynamic_oneToOne_allowUnlinked}
\end{figure}


Figures \ref{fig:insideLambda_uniform_phi0_nm400_dynamic_randomBipartite_allowUnlinked} and \ref{fig:insideLambda_uniform_phi0_nm400_dynamic_oneToOne_allowUnlinked} show statistical measures of selected values of $\lambda$, with Figures \ref{fig:fitting_insideLambda_uniform_phi0_nm400_dynamic_randomBipartite_allowUnlinked} and \ref{fig:fitting_insideLambda_uniform_phi0_nm400_dynamic_oneToOne_allowUnlinked} showing the fitting of the curve to a power law for a single selected value of $\lambda$.
A power law is appreciated in some of the plots but not all of them.
Tables \ref{tab:fitting_insideLambda_uniform_phi0_nm400_dynamic_randomBipartite_allowUnlinked} and \ref{tab:fitting_insideLambda_uniform_phi0_nm400_dynamic_oneToOne_allowUnlinked} show the values of the regression exponent and factor.
As with others, the single link initial condition is not plotted for selected values of $\lambda$ as it fails to evolve beyond the single link state.
\redtxt{TODO: descriure figures, no single link}
Appendix \ref{sec:app_figures_second-model} shows the same figures but with disconnected meanings disallowed.

\begin{figure}
  \centering
  \includegraphics[height=0.7\textheight]{insideLambda_uniform_phi0_nm400_dynamic_randomBipartite_allowUnlinked}
  \caption{Same information as in Figure \ref{fig:insideLambda_firstModel_phi0_nm400_dynamic_randomBipartite_allowUnlinked} but the graph follows the equations of the \secondmodel{} with $\pi$ following a uniform distribution. $\lambda^* = 0.4942$}
  \label{fig:insideLambda_uniform_phi0_nm400_dynamic_randomBipartite_allowUnlinked}
\end{figure}

\begin{figure}
  \centering
  \includegraphics[height=0.7\textheight]{insideLambda_uniform_phi0_nm400_dynamic_oneToOne_allowUnlinked.pdf}
  \caption{Same information as in Figure \ref{fig:insideLambda_firstModel_phi0_nm400_dynamic_randomBipartite_allowUnlinked} but the graph follows the equations of the \secondmodel{} with $\pi$ following a uniform distribution and the initial condition is one to one connections between words and meanings. $\lambda^* = 0.4970$}
  \label{fig:insideLambda_uniform_phi0_nm400_dynamic_oneToOne_allowUnlinked}
\end{figure}

\begin{figure}
  \centering
  \includegraphics[width=\textwidth]{fitting_insideLambda_uniform_phi0_nm400_dynamic_randomBipartite_allowUnlinked}
  \caption{Same information as in Figure \ref{fig:fitting_insideLambda_firstModel_phi0_nm400_dynamic_randomBipartite_allowUnlinked} but the model follows the equations of the \secondmodel{} with $\pi$ following a uniform distribution. $\lambda^*=0.4942$.
Table \ref{tab:fitting_insideLambda_uniform_phi0_nm400_dynamic_randomBipartite_allowUnlinked} shows the values of the exponent and the factor of the fitted power law.}
  \label{fig:fitting_insideLambda_uniform_phi0_nm400_dynamic_randomBipartite_allowUnlinked}
\end{figure}

\begin{figure}
  \centering
  \includegraphics[width=\textwidth]{fitting_insideLambda_uniform_phi0_nm400_dynamic_oneToOne_allowUnlinked}
  \caption{Same information as in Figure \ref{fig:fitting_insideLambda_firstModel_phi0_nm400_dynamic_randomBipartite_allowUnlinked} but the model follows the equations of the \secondmodel{} with $\pi$ following a uniform distribution and the initial condition is one to one connections between words and meanings. $\lambda^*=0.4970$.
Table \ref{tab:fitting_insideLambda_uniform_phi0_nm400_dynamic_oneToOne_allowUnlinked} shows the values of the exponent and the factor of the fitted power law.}
  \label{fig:fitting_insideLambda_uniform_phi0_nm400_dynamic_oneToOne_allowUnlinked}
\end{figure}

% latex table generated in R 4.1.1 by xtable 1.8-4 package
% Sat Oct  2 23:05:07 2021
\begin{table}[ht]
\centering
\begin{tabular}{lrr}
  \hline
plot & $\alpha$ & $k$ \\ 
  \hline
a & 2.6176507 & 0.6341345 \\ 
  b & 0.3812498 & 0.0021506 \\ 
  c & 1.1623482 & 37.4616016 \\ 
  d & -0.5600378 & 158.5629462 \\ 
  f & 1.1623482 & 37.4616016 \\ 
  g & 0.3954554 & 0.0250000 \\ 
   \hline
\end{tabular}
\caption{Table showing the exponent and factor of the power laws fitted in Figure \ref{fig:fitting_insideLambda_uniform_phi0_nm400_dynamic_randomBipartite_allowUnlinked}} 
\label{tab:fitting_insideLambda_uniform_phi0_nm400_dynamic_randomBipartite_allowUnlinked}
\end{table}


\begin{table}
  \centering
  \begin{adjustbox}{max width=\textwidth}
    \begin{tabular}{llSS}
      \toprule
      plot & law & $a$ & $k$ \\ 
      \midrule
      a & word frequency & 4.1719196 & 2.7845870 \\ 
      b & \redtxt{word frequency (cumulative)} & 0.2458167 & 0.0031418 \\ 
      c & meaning distribution & 4.0408414 & 717.6995365 \\ 
      d & meaning frequency & -0.9728453 & 272.7810669 \\ 
      f & \redtxt{meaning distribution} & 4.0408414 & 717.6995365 \\ 
      g & \redtxt{meaning distribution (cumulative)} & 0.2854165 & 0.0125000 \\ 
      \bottomrule
    \end{tabular}
  \end{adjustbox}
  \caption{\redtxt{en vermell: cal mostrar?} Table showing the exponent and factor of the power laws fitted in Figure \ref{fig:fitting_insideLambda_uniform_phi0_nm400_dynamic_oneToOne_allowUnlinked}} 
  \label{tab:fitting_insideLambda_uniform_phi0_nm400_dynamic_oneToOne_allowUnlinked}
\end{table}

%%% Local Variables:
%%% mode: latex
%%% TeX-master: "../tfm"
%%% End:


This last set of figures corresponds to the \secondmodel{} with $\phi=1$.
Figures \ref{fig:informationTheoretic_uniform_phi1_nm400_dynamic_randomBipartite_allowUnlinked},  \ref{fig:informationTheoretic_uniform_phi1_nm400_dynamic_singleLink_allowUnlinked} and \ref{fig:informationTheoretic_uniform_phi1_nm400_dynamic_oneToOne_allowUnlinked} show the information theoretic measures of the optimal graph for values of $\lambda$ ranging from 0 to 1.
\redtxt{TODO: descriure figures, phase transition at 0.5, disturbances just before 0.5}
Appendix \ref{sec:app_figures_second-model} also shows the same figures but with disconnected meanings disallowed.

\begin{figure}
  \centering
  \includegraphics[height=0.7\textheight]{informationTheoretic_uniform_phi1_nm400_dynamic_randomBipartite_allowUnlinked}
  \caption{Same information as in Figure \ref{fig:informationTheoretic_firstModel_phi0_nm400_dynamic_randomBipartite_allowUnlinked} but the graph uses the equations of the \secondmodel{} with $\pi$ following a uniform distribution and with $\phi=1$. Averages over 20 realizations.}
  \label{fig:informationTheoretic_uniform_phi1_nm400_dynamic_randomBipartite_allowUnlinked}
\end{figure}

\begin{figure}
  \centering
  \includegraphics[height=0.7\textheight]{informationTheoretic_uniform_phi1_nm400_dynamic_singleLink_allowUnlinked}
  \caption{Same information as in Figure \ref{fig:informationTheoretic_firstModel_phi0_nm400_dynamic_randomBipartite_allowUnlinked} but the graph uses the equations of the \secondmodel{} with $\pi$ following a uniform distribution with $\pi=1$.
The initial condition is a single link. Averages over 20 realizations.}
  \label{fig:informationTheoretic_uniform_phi1_nm400_dynamic_singleLink_allowUnlinked}
\end{figure}

\begin{figure}
  \centering
  \includegraphics[height=0.7\textheight]{informationTheoretic_uniform_phi1_nm400_dynamic_oneToOne_allowUnlinked}
  \caption{Same information as in Figure \ref{fig:informationTheoretic_firstModel_phi0_nm400_dynamic_randomBipartite_allowUnlinked} but the graph uses the equations of the \secondmodel{} with $\pi$ following a uniform distribution with $\pi=1$.
The initial condition is one to one connections between signals and meanings. Averages over 20 realizations.}
  \label{fig:informationTheoretic_uniform_phi1_nm400_dynamic_oneToOne_allowUnlinked}
\end{figure}

Figures \ref{fig:insideLambda_uniform_phi1_nm400_dynamic_randomBipartite_allowUnlinked} and \ref{fig:insideLambda_uniform_phi1_nm400_dynamic_oneToOne_allowUnlinked} show statistical measures of selected values of $\lambda$, with Figures \ref{fig:fitting_insideLambda_uniform_phi1_nm400_dynamic_randomBipartite_allowUnlinked} and \ref{fig:fitting_insideLambda_uniform_phi1_nm400_dynamic_oneToOne_allowUnlinked} showing the fitting of the curve to a power law for a single selected value of $\lambda$.
While not very exact, a behavior similar to a power law can be appreciated.
Tables \ref{tab:fitting_insideLambda_uniform_phi1_nm400_dynamic_randomBipartite_allowUnlinked} and \ref{tab:fitting_insideLambda_uniform_phi1_nm400_dynamic_oneToOne_allowUnlinked} show the values of the regression exponent and factor.
\redtxt{TODO: descriure figures, why no single link}
Appendix \ref{sec:app_figures_second-model} shows the same figures but with disconnected meanings disallowed.

\begin{figure}
  \centering
  \includegraphics[height=0.7\textheight]{insideLambda_uniform_phi1_nm400_dynamic_randomBipartite_allowUnlinked}
  \caption{Same information as in Figure \ref{fig:insideLambda_firstModel_phi0_nm400_dynamic_randomBipartite_allowUnlinked} but the graph follows the equations of the \secondmodel{} with $\phi=1$ with $\pi$ following a uniform distribution. $\lambda^* = 0.4992$.
  Averages over 20 realizations.}
  \label{fig:insideLambda_uniform_phi1_nm400_dynamic_randomBipartite_allowUnlinked}
\end{figure}

\begin{figure}
  \centering
  \includegraphics[height=0.7\textheight]{insideLambda_uniform_phi1_nm400_dynamic_oneToOne_allowUnlinked}
  \caption{Same information as in Figure \ref{fig:insideLambda_firstModel_phi0_nm400_dynamic_randomBipartite_allowUnlinked} but the graph follows the equations of the \secondmodel{} with $\pi$ following a uniform distribution and with $\phi=1$. The initial condition is one to one connections between words and meanings. $\lambda^* = 0.4991$.
Averages over 20 realizations.}
  \label{fig:insideLambda_uniform_phi1_nm400_dynamic_oneToOne_allowUnlinked}
\end{figure}

\begin{figure}
  \centering
  \includegraphics[width=\textwidth]{fitting_insideLambda_uniform_phi1_nm400_dynamic_randomBipartite_allowUnlinked}
  \caption{Same information as in Figure \ref{fig:fitting_insideLambda_firstModel_phi0_nm400_dynamic_randomBipartite_allowUnlinked} but the model follows the equations of the \secondmodel{} with $\pi$ following a uniform distribution and $\phi=1$. $\lambda^*=0.4992$.
Table \ref{tab:fitting_insideLambda_uniform_phi1_nm400_dynamic_randomBipartite_allowUnlinked} shows the values of the exponent and the factor of the fitted power law.}
  \label{fig:fitting_insideLambda_uniform_phi1_nm400_dynamic_randomBipartite_allowUnlinked}
\end{figure}

\begin{figure}
  \centering
  \includegraphics[width=\textwidth]{fitting_insideLambda_uniform_phi1_nm400_dynamic_oneToOne_allowUnlinked}
  \caption{Same information as in Figure \ref{fig:fitting_insideLambda_firstModel_phi0_nm400_dynamic_randomBipartite_allowUnlinked} but the model follows the equations of the \secondmodel{} with $\pi$ following a uniform distribution and $\phi=1$. The initial condition is one to one connections between words and meanings. $\lambda^*=0.4991$.
Table \ref{tab:fitting_insideLambda_uniform_phi1_nm400_dynamic_oneToOne_allowUnlinked} shows the values of the exponent and the factor of the fitted power law.}
  \label{fig:fitting_insideLambda_uniform_phi1_nm400_dynamic_oneToOne_allowUnlinked}
\end{figure}

\begin{table}
  \centering
  \begin{adjustbox}{max width=\textwidth}
    \begin{tabular}{llSS[scientific-notation=true]}
      \toprule
      plot & law & $a$ & $k$ \\ 
      \midrule
      a & word frequency ($\alpha$) & 2.6942979 & 0.7855283 \\ 
      %b & \redtxt{word frequency (cumulative)} & 0.3751754 & 0.0023094 \\ 
      b & meaning distribution ($\gamma$) & 2.6454013 & 229.8644521 \\ 
      c & meaning frequency ($\delta$) & -0.9825933 & 291.4160609 \\ 
      %f & \redtxt{meaning distribution} & 2.6454013 & 229.8644521 \\ 
      %g & \redtxt{meaning distribution (cumulative)} & 0.4641743 & 0.0200000 \\ 
      \bottomrule
    \end{tabular}
  \end{adjustbox}
  \caption{Table showing the exponent and factor of the power laws fitted for the \secondmodel{} with $\phi=1$ and a random bipartite graph as the initial condition (Figure \ref{fig:fitting_insideLambda_uniform_phi1_nm400_dynamic_randomBipartite_allowUnlinked})
    In this table, $\alpha \approx 2.7$, $\delta \approx 1$ and $\gamma \approx 2.7$.
    The relationship between these values (Equation \eqref{eq:relation-exponents}) holds but the exponents are not exactly the ones expected.
  }
  \label{tab:fitting_insideLambda_uniform_phi1_nm400_dynamic_randomBipartite_allowUnlinked}
\end{table}

%%% Local Variables:
%%% mode: latex
%%% TeX-master: "../tfm"
%%% End:


\begin{table}
  \centering
  \begin{adjustbox}{max width=\textwidth}
    \begin{tabular}{llSS[scientific-notation=true]}
      \toprule
      plot & law & $a$ & $k$ \\ 
      \midrule
      a & word frequency ($\alpha$) & 3.8860954 & 2.27097983 \\ 
      %b & \redtxt{word frequency (cumulative)} & 0.2671876 & 0.00303420 \\ 
      b & meaning distribution ($\gamma$) & 3.8007579 & 632.05935245 \\ 
      c & meaning frequency ($\delta$) & -0.9804221 & 293.30509675 \\ 
      %f & \redtxt{meaning distribution} & 3.8007579 & 632.05935245 \\ 
      %g & \redtxt{meaning distribution (cumulative)} & 0.2924813 & 0.01500000 \\ 
      \bottomrule
    \end{tabular}
  \end{adjustbox}
  \caption{Table showing the exponent and factor of the power laws fitted for the \secondmodel{} with $\phi=1$ and a one to one configuration as the initial condition (Figure \ref{fig:fitting_insideLambda_uniform_phi1_nm400_dynamic_oneToOne_allowUnlinked})
    In this table, $\alpha \approx 3.9$, $\delta \approx 1$ and $\gamma \approx 3.9$.
    The relationship between these values (Equation \eqref{eq:relation-exponents}) holds but the exponents are not exactly the ones expected.
  } 
  \label{tab:fitting_insideLambda_uniform_phi1_nm400_dynamic_oneToOne_allowUnlinked}
\end{table}

%%% Local Variables:
%%% mode: latex
%%% TeX-master: "../tfm"
%%% End:


%%% Local Variables:
%%% mode: latex
%%% TeX-master: "tfm"
%%% End:
