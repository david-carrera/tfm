\chapter{Introduction}
Aquesta part es important, ha de quedar clara
\begin{itemize}
\item marc motivació
  \begin{itemize}
  \item model existent pot explicar mes lleis
  \end{itemize}
\item patrons llengues
\end{itemize}
\section{State of the art / background / ...}
\begin{itemize}
\item ...
\end{itemize}
\section{Objectius generals}
\begin{itemize}
\item quines lleis linguistiques es reprodueixen
\item ...
\end{itemize}
\section{intro model}
\begin{itemize}
\item teoria informació
  \begin{itemize}
  \item entropi
  \item informació mutua
  \item formules
  \end{itemize}
\item graf bipartit
  \begin{itemize}
  \item conceptes del graf bipartit
  \item sense matematiques
  \item no hi ha arestes entre vertexos del mateix grup
  \item grups: paraules i significats
  \end{itemize}
\item intro primer model
  \begin{itemize}
  \item conceptes del primer model
  \item sense matematiques
  \item probabilitat d'una paraula depen del nombre de significats
  \item probabilitat d'un significat depen del nombre de paraules
  \end{itemize}
\item intro segon model
  \begin{itemize}
  \item conceptes del segon model
  \item sense matematiques
  \item probabilitat de significats depen de probabilitat a priori
  \end{itemize}
\item parametre phi
\end{itemize}
\section{Objectius}
\begin{itemize}
\item reproduir resultats antics
\item crear eina codi obert
  \begin{itemize}
  \item pot reproduir resultats antics, resolent problemes de replicabilitat
  \item pot crear nous resultats amb varis parametres
  \end{itemize}
\item optimització: minims locals?
\item ...
\end{itemize}
\section{Hipòtesis}
\begin{itemize}
\item article vitevith dona una relacio entre num significats
\item lleis linguistiques surten de la optimitzacio
\item negligim interacció social (unlike e.g. naming game)
\item la optimitzacio no arriba necessariament a un optim global
\end{itemize}
\section{problemes}
\subsection{llinguistica quantitativa}
\begin{itemize}
\item que prediuen els models actuals
\end{itemize}
\subsection{computacional}
\begin{itemize}
\item calcular omega eficientment
\item asegurar correctesa (tests)
\item condicio aturada optimitzacio
\end{itemize}
\subsection{calcul dinamic}
\begin{itemize}
\item complexitat matematica
\item error numeric
\end{itemize}
\subsection{outline of the thesis}
\begin{itemize}
\item cadascuna de les parts de la tesis
\item mencionar el que al final s'explica a la discussió
\end{itemize}

En algun lloc de la intro: Taula de resum dels models.
\begin{itemize}
\item Columnes: Ramon typing, Simon's model, Model 2003 amb phi = 0, Model 2005 amb phi = 0, Model 2003 amb phi = 1, Model 2005 amb phi = 1.
\item Rengles: Zipf rank-frequency law, Zipf's law of meaning distribution, Zipf's law of the age of words, vocabulary learning bias.
\item Cel·les: Yes, No, Aclariments.
\end{itemize}

%%% Local Variables:
%%% mode: latex
%%% TeX-master: "tfm"
%%% End:
