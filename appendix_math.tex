\chapter{Formulae}
\label{cha:app_formulae}

\redtxt{derivacions llargues i pagines i pagines de mates van aqui}

\section{Properties}

\subsection{Simplification of the equations of entropies}
\label{sec:app_formulae_trick}

Property \ref{th:definition-trick} is used in several parts of the thesis to simplify the expressions of entropies.
The derivation is short, but it is not added into the main text of the thesis for the sake of clarity and focus.

\begin{align*}
              -\sum_i & \frac{x_i}{T} \log\frac{x_i}{T} \\
  -\frac{1}{T} \sum_i & x_i \log x_i - x_i \log T \\
  -\frac{1}{T} \sum_i & x_i \log x_i + \log(T) \frac{1}{T} \sum_i x_i \\
\end{align*}
at this point we can apply
\begin{equation*}
  \sum_i x_i = T
\end{equation*}
and obtain
\begin{align*}
  \log T - \frac{1}{T} \sum_i & x_i \log x_i\\
\end{align*}

\subsection{Proof that $\sum_{i=1}^n \mu_i^\phi \chi_i = \rho$}
\label{sec:app_formulae_proof-sum-HS_second-model}

In order to use Property \ref{th:definition-trick} to simplify the derivation of Equation \eqref{eq:definition-HS_second-model}, Property \ref{th:sum-mui-chii-equals-rho} must hold
Here it is shown that this is the case.

Starting from the equality, we can expand both sides
\begin{align*}
\sum_{i=1}^n \mu_i^\phi \chi_i =& \rho \\
\sum_{i=1}^n \mu_i^\phi \sum_{j=1}^m \frac{a_{ij} (1 - \delta_{\omega_j,0}) \pi(r_j)}{\omega_{\phi,j}} =& \sum_{j=1}^m (1 - \delta_{\omega_j,0}) \pi(r_j) \\
\sum_{j=1}^m \left( \sum_{i=1}^n a_{ij} \mu_i^\phi \right) \frac{(1 - \delta_{\omega_j,0}) \pi(r_j)}{\omega_{\phi,j}} =& \sum_{j=1}^m (1 - \delta_{\omega_j,0}) \pi(r_j), \\
\end{align*}
applying Equation \eqref{eq:definition-omegaphi} we obtain
\begin{equation*}
  \sum_{j=1}^m (1 - \delta_{\omega_j,0}) \pi(r_j)
\end{equation*}
on both sides.

Recall the definition of $p(s_i)$ in Equation \eqref{eq:definition-psi_second-model} to see that
\begin{equation*}
  \sum_{i=1}^n p(s_i) = 1
\end{equation*}
immediately follows.

\section{Derivation of the joint entropy in the \secondmodel}
\label{sec:app_formulae_join-entropy_second-model}

In Section \ref{sec:model_math_second-model} the derivation of the joint entropy is left as just the high level reasoning of which formulas may be used to obtain its expression.
Here, the full derivation of both approaches is given.

The first approach consists of applying Equation \eqref{eq:definition-join-prob_second-model} to the information theory definition of $H(S,R)$ (Equation \eqref{eq:definition-HSR}),
\begin{align*}
  H(S,R) =& - \sum_{i=1}^n \sum_{j=1}^m p(s_i, r_j) \log \left[ p(s_i, r_j) \right] \\
         =& - \sum_{i=1}^n \sum_{j=1}^m \frac{a_{i,j} (1 - \delta_{\omega_j,0}) \mu_i^\phi \pi(r_j)}{\rho \omega_{\phi,j}} \log \left[ \frac{a_{i,j} (1 - \delta_{\omega_j,0}) \mu_i^\phi \pi(r_j)}{\rho \omega_{\phi,j}} \right] \\
         =& - \frac{1}{\rho} \sum_{i=1}^n \sum_{j=1}^m \frac{a_{i,j} (1 - \delta_{\omega_j,0}) \mu_i^\phi \pi(r_j)}{\omega_{\phi,j}} \log \left[ \frac{\mu_i^\phi \pi(r_j)}{\rho \omega_{\phi,j}} \right] \\
         =& - \frac{1}{\rho} \sum_{j=1}^m \frac{(1 - \delta_{\omega_j,0}) \pi(r_j)}{\omega_{\phi,j}} \Bigg[ \phi \sum_{i=1}^n a_{i,j} \mu_i^\phi \log \mu_i \\
          & + \log \pi(r_j) \sum_{i=1}^n a_{i,j} \mu_i^\phi - \log(\rho) \sum_{i=1}^n a_{i,j} \mu_i^\phi \\
          & - \log(\omega_{\phi,j}) \sum_{i=1}^n a_{i,j} \mu_i^\phi \Bigg].
\end{align*}
Applying Equation \eqref{eq:definition-nu} we arrive at Equation \eqref{eq:definition-HSR_second-model}.

The second approach consists on using other less complex expressions we have available and building up from them using the definitions
\begin{align}
  \label{eq:definition-HSR_appendix}
  H(S,R) &= H(S|R) + H(R), \\
  \label{eq:definition-HSgR_appendix}
  H(S|R) &= \sum_{j=1}^m H(S|r_j)p(r_j), \\
  \label{eq:definition-HSgrj_appendix}
  H(S|r_j) &= -\sum_{i=1}^n p(s_i|r_j) \log p(s_i|r_j).
\end{align}

Starting from Equation \eqref{eq:definition-HSgrj_appendix} and applying Equation \eqref{eq:definition-cond-prob_second-model} we obtain
\begin{align*}
  H(S|r_j) =& -\sum_{i=1}^n \frac{a_{i,j} \mu_i^\phi \pi(r_j)}{\rho \omega_{\phi,j}} \log \left( \frac{a_{i,j} \mu_i^\phi \pi(r_j)}{\rho \omega_{\phi,j}} \right) \\
           =& \log \omega_{\phi,j} - \frac{1}{\omega_{\phi,j}} \sum_{i=1}^n a_{i,j} \mu_i^\phi \log(a_{i,j} \mu_i^\phi) \\
           =& \log \omega_{\phi,j} - \frac{\phi}{\omega_{\phi,j}} \sum_{i=1}^n a_{i,j} \mu_i^\phi \log(\mu_i)
\end{align*}
and after applying Equation \eqref{eq:definition-nu}
\begin{equation}
  \label{eq:definition-HSgrj_appendix-step}
  H(S|r_j) = \log \omega_{\phi,j} - \frac{\phi \nu_j}{\omega_{\phi,j}}.
\end{equation}

Applying Equations \eqref{eq:definition-HSgrj_appendix-step} and \eqref{eq:definition-prj_second-model} to Equation \eqref{eq:definition-HSgR_appendix} we reach
\begin{align}
  H(S|R) =& \sum_{j=1}^m \left(\log \omega_{\phi,j} - \frac{\phi \nu_j}{\omega_{\phi,j}}\right) \frac{(1 - \delta_{w_j,0}) \pi(r_j)}{\rho} \nonumber \\
  \label{eq:definition-HSgR_appendix-step}
       =& \frac{1}{\rho} \sum_{j=1}^m (1 - \delta_{w_j,0}) \pi(r_j) \left[ \log(\omega_{\phi,j}) - \frac{\phi \nu_j}{\omega_{\phi,j}} \right].
\end{align}

The last step consists of applying Equations \eqref{eq:definition-HSgR_appendix-step} and \eqref{eq:definition-HR_second-model} to Equation \eqref{eq:definition-HSR_appendix}, obtaining Equation \eqref{eq:definition-HSR_second-model} again
\begin{align*}
  H(S,R) &= \frac{1}{\rho} \sum_{j=1}^m (1 - \delta_{w_j,0}) \pi(r_j) \left[ \log(\omega_{\phi,j}) - \frac{\phi \nu_j}{\omega_{\phi,j}} \right] \\
       &+ \log \rho - \frac{1}{\rho} \sum_{j=1}^m (1 - \delta_{w_j,0}) \pi(r_j) \log \pi(r_j) \\
       &= \log \rho - \frac{1}{\rho} \sum_{j=1}^m (1 - \delta_{w_j,0}) \pi(r_j) \left[ \frac{\phi \nu_j}{\omega_{\phi,j}} + \log \frac{\pi(r_j)}{\omega_{\phi,j}} \right].
\end{align*}

\section{Dynamic Equations for the \secondmodel{}}
\label{sec:app_formulae_dynamic-equations_second-model}

The full derivation of the dynamic equations for the \secondmodel{} is given in this section.
The equations themselves as well as other definitions used here are already given in Section \ref{sec:model_math_second-model}.

In this section, the derivation is logical rather than mathematical, consisting of explanations behind the logic of the equations.

$\rho$ (see Equation \eqref{eq:definition-rho}) only changes when, as a result of the mutation, the meaning $r_j$ either was connected and becomes disconnected or was disconnected and becomes connected, resulting in Equation \eqref{eq:definition-rho_dynamic}.

$\nu_l$ is very similar to $\omega_{\phi,l}$ (compare Equations \eqref{eq:definition-nu} and \eqref{eq:definition-omegaphi}).
Equation \eqref{eq:definition-nu_dynamic} shows how $\nu_l$ changes in the same cases and in the same way as $\omega_{\phi,l}$ (Equation \eqref{eq:definition-wphik_dynamic}).

$\chi_k$ depends on  $\omega_{\phi,l}$ such that $l \in \Gamma_S(k)$ (see Equation \eqref{eq:definition-chi_second-model}).
For $\chi_i$, the entire value has to be recalculated, as every $\omega_{\phi,l}$ for $l \in \Gamma_S(k)$ will have changed (Equation \eqref{eq:definition-wphik_dynamic}).
It is more efficient (and reduces the amount of floating point error) to calculate $\chi_i$ statically than to subtract every $\omega_{\phi,l}$ and then add every $\omega'_{\phi,l}$.
For all other values of $\chi_k$ ($k \neq i$), the affected $\omega_{\phi,l}$ are the intersection of the set of neighbors of $i$ plus the meaning $r_j$ and the set of neighbors of $k$. This is the set $A_{i,j}(k)$ defined in Equation \eqref{eq:definition-A_second-model}. With this reasoning we reach the formula for the dynamic recalculation of $\chi_k$, Equation \eqref{eq:definition-chi_dynamic}.

$X(R)$ (Equation \eqref{eq:definition-XR_second-model}) changes in a very similar way to $\rho$ (Equation \eqref{eq:definition-rho}), as they both depend on the same variable in the same way: only when the meaning $r_j$ becomes connected or disconnected does $X(R)$ change, resulting in Equation \eqref{eq:definition-rho_dynamic}.

$X(S,R)$ (Equation \eqref{eq:definition-XSR_second-model}) will change when a $x(r_l)$ (Equation \eqref{eq:definition-xrj_second-model}) changes.
Both $\omega_{\phi,l}$ and $\nu_l$ change for $l \in \Gamma_S(i) \cup \{j\}$ (Equations \eqref{eq:definition-wphik_dynamic} and \eqref{eq:definition-nu_dynamic}).
So these are the components $x(s_l)$ of $H(S,R)$ that will change.
For the special case of $l=j$ we should not subtract the old value when $r_j$ has become connected as a result of the mutation (as that component was not present), and we should not add the new value when $r_j$ has become disconnected (as that component should not be present).
Or in other words, only subtract the old value if $r_j$ was connected before the mutation (this serves to either update it or to remove it) and only add the new value if $r_j$ is connected after the mutation (this serves to either update it or to add it).
The resulting expression for $X'(S,R)$ is in Equation \eqref{eq:definition-XSR_second-model_dynamic}.

$X(S)$ (Equation \eqref{eq:definition-XS_second-model}) will change when $x(s_k)$ (Equation \eqref{eq:definition-xsi_second-model}) changes, which depends on $\chi_k$ (Equation \eqref{eq:definition-chi_dynamic}).
$\chi_k$ is updated for every $k$ such that $|A_k| \neq \emptyset$ and $\chi_i$ is always updated, which is the definition of the $B_{i,j}(k)$ set (Equation \eqref{eq:definition-B_second-model}).
Using $B_{i,j}(k)$ we obtain the expression for $X'(S)$ in Equation \eqref{eq:definition-XS_second-model_dynamic}.

%%% Local Variables:
%%% mode: latex
%%% TeX-master: "tfm"
%%% End:
