\chapter{Model}
\label{cha:model}

\redtxt{TODO: reasoning of global optimum that is used to stop early}

This chapter covers the details about the two models previously introduced in Chapter \ref{cha:introduction}, where a high level introduction to both the \firstmodel{} and the \secondmodel{} was given.
It follows a top down approach.
General concepts from both types of model are outlined first and the specific details from each model are given afterwards.

As a reminder of the previous chapter a very brief definition of the two models follows.
Both models are quite similar, they represent a way to connect words with meanings.
Both meanings and words are given probabilities of being used, from which information theoretic measures are derived, which are ultimately used to obtain a cost function.
The high level differentiating trait is the way in which the probability of a meaning is obtained.
In the first studied model (\firstmodel{}), the probabilities of both words and meanings are \emph{internal} to the model, they only depend on the structure of the bipartite graph.
In the second model (\secondmodel{}), the probabilities of meanings are \emph{external} to the model, they come from a distribution of \emph{a priori} probabilities.

The remainder of this chapter is divided into two sections.
Section \ref{sec:model_math} deals with the mathematical definitions of the models and it is more theoretical in nature.
Section \ref{sec:model_compute} outlines the computational aspects of the two models and it is more practical and implementation oriented.
This chapter does not deal with any actual details of the implementation of the model.
These details are given in Chapter \ref{cha:methods}.

Section \ref{sec:introduction_model} from Chapter \ref{cha:introduction} introduces the  the graph theory and information theory concepts that will appear during the rest of this chapter.

\section{Mathematical aspect}
\label{sec:model_math}

This section covers the mathematical definition of the model.
Section \ref{sec:model_math_graph} introduces concepts common to both models.
These concepts are then extended for either the \firstmodel{} or the \secondmodel{} in Sections \ref{sec:model_math_first-model} and \ref{sec:model_math_second-model} respectively.

\subsection{Common concepts}
\label{sec:model_math_graph}

Here a few more common concepts that were not discussed in Chapter \ref{cha:introduction} are introduced.

As a reminder, the notation defined for the degree of the vertices in the graph is repeated.
The word $s_i$ has degree $\mu_i$, while the meaning $r_j$ has degree $\omega_j$.
$\mu$ and $\omega$ are defined as
\begin{equation}
  \label{eq:definition-mu}
  \mu_i = \sum_{j=1}^m a_{i,j}
\end{equation}
and
\begin{equation}
  \label{eq:definition-omega}
  \omega_j = \sum_{i=1}^n a_{i,j}.
\end{equation}

There is an additional parameter of the models, $\phi$, which is not directly related to the bipartite graph.
This parameter appears in both models and it is used to ``emphasize'' the effect of the vertex degrees in the calculations.
As explained in previous sections, this parameter is a new addition to the models originally presented in \cite{Ferrer2005a} and \cite{Ferrer2003a}.

With the parameter $\phi$, the definitions of $\mu$ and $\omega$ can be generalized.
$\mu_\phi$ and $\omega_\phi$ are defined as
\begin{equation}
  \label{eq:definition-muphi}
  \mu_{\phi,i} = \sum_{j=1}^m a_{i,j} \omega_j^\phi
\end{equation}
and
\begin{equation}
  \label{eq:definition-omegaphi}
  \omega_{\phi,i} = \sum_{i=1}^n a_{i,j} \mu_i^\phi.
\end{equation}
It can be seen that when $\phi=0$ Equation \eqref{eq:definition-muphi} becomes Equation \eqref{eq:definition-mu} and Equation \eqref{eq:definition-omegaphi} becomes Equation \eqref{eq:definition-omega}.
In other words, $\mu_{0,i} = \mu_i$ and $\omega_{0,j} = \omega_j$

\subsection{The \firstmodel{}}
\label{sec:model_math_first-model}

Here the concepts outlined in Section \ref{sec:introduction_model_first-model} are worked out into actual probabilities and information theoretic equations.

This model is defined by the joint probability of a word $s_i$ and a meaning $r_j$ seen in Equation \eqref{eq:psirj-proportional-mui-wj}.
As seen in Section \ref{sec:introduction_model_first-model}, this probability must be proportional to the product of the degrees of $s_i$ and $r_j$.

From this definition equations for the marginal probabilities are derived, which in turn are used to obtain the information theory expressions in the cost function $\Omega$ (Equation \eqref{eq:definition-Omega}).
From these expressions, dynamic equations are derived to obtain the change experienced by the entropies after a single mutation has occurred on the adjacency matrix $A$.

\subsubsection{Joint Probability}

Adding a normalizing factor $M_\phi$, the joint probability becomes
\begin{equation}
  \label{eq:definition-psirj_first-model}
  p(s_i, r_j) = \frac{1}{M_\phi} a_{i,j} (\mu_i \omega_j)^\phi.
\end{equation}

This normalizing factor is obtained by applying the definition of probability
\begin{equation*}
  \sum_{i=1}^n \sum_{j=1}^m p(s_i, r_j) = 1.
\end{equation*}
Then,
\begin{equation*}
  M_\phi = \sum_{i=1}^n \sum_{j=1}^m a_{i,j} (\mu_i \omega_j)^\phi
\end{equation*}
or equivalently
\begin{equation}
  \label{eq:definition-Mphi}
  M_\phi = \sum_{(i,j) \in E}^n (\mu_i \omega_j)^\phi.
\end{equation}
Notably, when $\phi=0$, $M_\phi$ is simply $M$ and as follows from Equation \eqref{eq:definition-Mphi}, it's the total number of edges in the graph.

\subsubsection{Marginal Probabilities}

Formulas for the marginal probabilities $p(s_i)$ and $p(r_j)$ are derived from the joint probability using the general formula
\begin{equation}
  \label{eq:marginal-from-joint}
  p(x) = \sum_{y \in Y} p(x,y).
\end{equation}
Applying Equation \eqref{eq:marginal-from-joint},
\begin{equation*}
  p(s_i) = \frac{\mu_i^\phi}{M_\phi} \sum_{j=1}^m a_{i,j} \omega_j^\phi.
\end{equation*}
Recall Equation \eqref{eq:definition-muphi},
\begin{equation}
  \label{eq:definition-psi_first-model}
  p(s_i) = \frac{\mu_i^\phi \mu_{\phi,i}}{M_\phi}.
\end{equation}
Symmetrically applying Equation \eqref{eq:marginal-from-joint} to obtain $p(r_j)$ and applying Equation \eqref{eq:definition-omegaphi} instead we obtain
\begin{equation}
  \label{eq:definition-prj_first-model}
  p(r_j) = \frac{\omega_j^\phi \omega_{\phi,j}}{M_\phi}.
\end{equation}

\subsubsection{Entropies}

$H(S,R)$ is obtained by applying the joint probability given in Equation \eqref{eq:definition-psirj_first-model} to Equation \eqref{eq:definition-HSR},
\begin{equation}
  \label{eq:step-HSR}
  H(S,R) = -\sum_{i=1}^n \sum_{j=1}^m \frac{1}{M_\phi} a_{i,j} (\mu_i \omega_j)^\phi \log \left( a_{i,j} (\mu_i \omega_j)^\phi \right).
\end{equation}

\begin{property}{}{definition-trick}
  This expression can be refined taking advantage of the equality
  \begin{equation*}
    -\sum_i \frac{x_i}{T} \log\frac{x_i}{T} = \log T - \frac{1}{T} \sum_i x_i \log x_i.
  \end{equation*}
  This equality holds as long as
  \begin{equation*}
    T = \sum_i x_i.
  \end{equation*}
  While is quite simple to prove that this is true, the derivation is given in
  Appendix \ref{sec:app_formulae_trick}.

\end{property}

Applying Property \ref{th:definition-trick} to Equation \eqref{eq:step-HSR} we obtain
\begin{equation*}
  H(S,R) = \log M_\phi - \frac{\phi}{M_\phi} \sum_{i=1}^n \sum_{j=1}^m a_{i,j} \left( (\mu_i \omega_j)^\phi \log a_{i,j} \mu_i \omega_j \right)
\end{equation*}
using $E$ in the summation and adopting the convention that $0 \log 0 = 0$
\begin{equation}
  \label{eq:definition-HSR_first-model}
  H(S,R) = \log M_\phi - \frac{\phi}{M_\phi} \sum_{(i,j) \in E} (\mu_i \omega_j)^\phi \log \mu_i \omega_j
\end{equation}
is reached.

The marginal word entropy is found applying the marginal probability in Equation \eqref{eq:definition-psi_first-model} to the definition of $H(S)$ in Equation \eqref{eq:definition-HS}
\begin{equation*}
  H(S) = -\sum_{i=1}^n \frac{\mu_i \mu_{\phi,i}}{M_\phi} \log \frac{\mu_i \mu_{\phi,i}}{M_\phi}.
\end{equation*}
Property \ref{th:definition-trick} can be applied immediately, obtaining
\begin{equation}
  \label{eq:definition-HS_first-model}
  H(S) = \log M_\phi - \frac{1}{M_\phi} \sum_{i=1}^n \mu_i^\phi \mu_{\phi,i} \log \mu_i^\phi \mu_{\phi,i}.
\end{equation}
The convention $0 \log 0 = 0$ is applied here for disconnected words (where $\mu_i = 0$).

The marginal meaning entropy is found symmetrically, the marginal probability (Equation \eqref{eq:definition-prj_first-model}) is applied to the definition of $H(R)$ (Equation \eqref{eq:definition-HR}).
With Property \eqref{th:definition-trick}, we obtain
\begin{equation}
  \label{eq:definition-HR_first-model}
  H(R) = \log M_\phi - \frac{1}{M_\phi} \sum_{j=1}^m \omega_j^\phi \omega_{\phi,j} \log \omega_j^\phi \omega_{\phi,j}.
\end{equation}
also using the convention $0 \log 0 = 0$ for disconnected meanings ($\omega_j=0$).

\subsubsection{Dynamic Equations}

The full derivation of the dynamic equations is given in \cite{Carrera2021a}.
Here only the final expressions of the dynamic equations are given.

Starting with compact expressions of the entropies (Equations \eqref{eq:definition-HSR_first-model}, \eqref{eq:definition-HS_first-model} and \eqref{eq:definition-HR_first-model})
\begin{align}
  \label{eq:definition-HSR_first-model_compact}
  H(S,R) &= \log M_\phi - \frac{\phi}{M_\phi} X(S,R) \\
  \label{eq:definition-HS_first-model_compact}
  H(S) &= \log M_\phi - \frac{1}{M_\phi} X(S) \\
  \label{eq:definition-HR_first-model_compact}
  H(R) &= \log M_\phi - \frac{1}{M_\phi} X(R)
\end{align}
with
\begin{align}
  \label{eq:definition-XSR_first-model_dynamic}
  X(S,R) &= \sum_{(i,j) \in E} x(s_i, r_j) \\
  \label{eq:definition-XS_first-model_dynamic}
  X(S) &= \sum_{i=1}^n x(s_i) \\
  \label{eq:definition-XR_first-model_dynamic}
  X(R) &= \sum_{j=1}^m x(r_j) \\
  \label{eq:definition-Xsirj_first-model_dynamic}
  x(s_i, r_j) &= (\mu_i\omega_j)^\phi \log \mu_i\omega_j \\
  \label{eq:definition-Xsi_first-model_dynamic}
  x(s_i) &= \mu_i^\phi \mu_{\phi,i} \log \mu_i^\phi \mu_{\phi,i} \\
  \label{eq:definition-Xrj_first-model_dynamic}
  x(r_j) &= \omega_j^\phi \omega_{\phi,j} \log \omega_j^\phi \omega_{\phi,j}.
\end{align}

A prime mark is used to indicate a new value of a certain variable after a mutation has taken place.
A variable without a prime mark indicates the value before the mutation took place.
Suppose that $a_{i,j}$ mutates.
Then
\begin{align}
  \label{eq:definition-aij_dynamic}
  a'_{i,j} &= 1 - a_{i,j} \\
  \label{eq:definition-mui_dynamic}
  \mu'_i &= \mu_i + (-1)^{a_{i,j}} \\
  \label{eq:definition-wj_dynamic}
  \omega'_j &= \omega_j + (-1)^{a_{i,j}}
\end{align}

We define the set of neighbors of any word $s_i$
\begin{equation}
  \label{eq:definition-Gamma-S}
  \Gamma_S(i) = \Set{r_j}{(s_i,r_j) \in E},
\end{equation}
and similarly the set of neighbors of any meaning $r_j$
\begin{equation}
  \label{eq:definition-Gamma-R}
  \Gamma_R(j) = \Set{s_i}{(s_i,r_j) \in E}.
\end{equation}

Then, for any $k$ such that $1 \leq k \leq n$, we have that
\begin{equation}
  \label{eq:definition-muphik_dynamic}
  \mu'_{\phi,k} = \begin{cases}
    \mu_{\phi,k} - a_{ij} \omega_j^\phi + (1 - a_{ij}) {\omega_j'}^\phi & \text{if}~k=i \\
    \mu_{\phi,k} - \omega_j^\phi + {\omega_j'}^\phi & \text{if}~k \in \Gamma_R(j)~\text{and}~k \neq i \\
    \mu_{\phi,k} & \text{otherwise}.
  \end{cases}
\end{equation}
Likewise, for any $l$ such that $1 \leq l \leq m$, we have that
\begin{equation}
  \label{eq:definition-wphik_dynamic}
  \omega'_{\phi,l} = \begin{cases}
    \omega_{\phi,l} - a_{ij} \mu_i^\phi + (1 - a_{ij}) {\mu_i'}^\phi & \text{if}~l=j \\
    \omega_{\phi,l} - \mu_i^\phi + {\mu_i'}^\phi & \text{if}~l \in \Gamma_S(i)~\text{and}~l \neq j \\
    \omega_{\phi,l} & \text{otherwise}.
  \end{cases}.
\end{equation}
These variables can be applied to Equations \eqref{eq:definition-Xsirj_first-model_dynamic}, \eqref{eq:definition-Xsi_first-model_dynamic} and \eqref{eq:definition-Xrj_first-model_dynamic} directly to obtain their values.

We define the set $E_{i,j}$ as the set of all edges connecting $s_i$ and $r_j$ with their neighbors.
That is,
\begin{equation}
  \label{eq:definition-Nij}
  E_{i,j} = \Set{(i,l)}{l \in \Gamma_S(i)} \cup \Set{(k,j)}{k \in \Gamma_R(j)}
\end{equation}

With these definitions, we can now obtain the expressions of $X'(S,R)$ from $X(S,R)$ and of $M'_\phi$ from $M_\phi$.
\begin{equation*}
\begin{split}
  M'_{\phi} = M_\phi &- \left[ \sum_{(k,l) \in E_{i,j}}(\mu_k \omega_l)^\phi \right]   - a_{ij} (\mu_i \omega_j)^\phi \\
                     &+ \left[ \sum_{(k,l) \in E_{i,j}}(\mu'_k \omega'_l)^\phi \right] + (1 - a_{ij}) (\mu'_i \omega'_j)^\phi.
\end{split}
\end{equation*}
Similarly, the new value of $X(S,R)$ will be
\begin{equation}
  \label{eq:definition-XSR_first-model_dynamic}
\begin{split}
  X'(S,R) = X(S,R) &- \left[ \sum_{(k,l) \in E_{i,j}}x(s_k, r_l) \right] - a_{ij} x(s_i, r_j) \\
                   &+ \left[ \sum_{(k,l) \in E_{i,j}}x'(s_k, r_l)\right] + (1 - a_{ij}) x'(s_i, r_j).
\end{split}
\end{equation}

The expressions of $X'(S)$ from $X(S)$ and $X'(R)$ from $X(R)$ used to dynamically update Equations \eqref{eq:definition-XSR_first-model_dynamic}, \eqref{eq:definition-XS_first-model_dynamic} and \eqref{eq:definition-XR_first-model_dynamic} are
\begin{equation}
  \label{eq:definition-XS_first-model_dynamic}
  X'(S) = X(S) - \left[ \sum_{k \in \Gamma_{R}(j)} x(s_k) \right] - a_{ij} x(s_i) + \left[ \sum_{k \in \Gamma_{R}(j)} x'(s_k) \right] + (1 - a_{ij}) x'(s_i)
\end{equation}
and
\begin{equation}
  \label{eq:definition-XR_first-model_dynamic}
  X'(R) = X(R) - \left[ \sum_{l \in \Gamma_{S}(i)} x(r_l) \right] - a_{ij} x(r_j) + \left[ \sum_{l \in \Gamma_{S}(i)} x'(r_l) \right] + (1 - a_{ij}) x'(r_j).
\end{equation}

Again, the full derivation of these equations is given in \cite{Carrera2021a} and it is not included here to avoid repeating the same ideas.

The values of $H(S)$, $H(R)$ and $H(S,R)$ can be obtained by applying Equations \eqref{eq:definition-XSR_first-model_dynamic}, \eqref{eq:definition-XS_first-model_dynamic} and \eqref{eq:definition-XR_first-model_dynamic} to the definitions in Equations \eqref{eq:definition-HSR_first-model_compact}, \eqref{eq:definition-HS_first-model_compact} and \eqref{eq:definition-HR_first-model_compact} respectively.

\subsubsection{Extreme cases and invariants}
\label{sec:model_math_first-model_invariant}

For verification purposes (see Section \ref{sec:methods_verification}), the values of the entropies along with their invariants are also given here.

\paragraph{Extreme cases} These cases can be easily derived from Equations \eqref{eq:definition-HSR_first-model}, \eqref{eq:definition-HS_first-model} and \eqref{eq:definition-HR_first-model} ($H(S,R)$, $H(S)$ and $H(R)$ respectively) by assuming the corresponding condition and applying elementary algebra.

\begin{itemize}
\item For a single edge, $H(S), H(R), H(S,R) = 0$.
\item For a complete graph, $H(S) = \log n, H(R) = \log m, H(S,R) = \log nm $
\item For a one-to-one mapping of signals into meanings with $n=m$, $H(S), H(R), H(S,R) = \log n$
\end{itemize}

\paragraph{Invariants} These invariants follow from basic information theory.
\cite{Cover1999}

\begin{itemize}
\item $0 \leq H(S) \leq \log n$
\item $0 \leq H(R) \leq \log m$
\item $0 \leq H(S,R) \leq \log nm$
\end{itemize}

\subsection{The \secondmodel{}}
\label{sec:model_math_second-model}

Here the equations and concepts form \ref{sec:introduction_model_second-model} are worked out into probabilities and information theoretic equations.

In this model, $p(r_j)$ depends in an \emph{a priori} probability $\pi(r_j)$ while $p(s_i | r_j)$ is defined in terms of $\mu_i^\phi.$

From the definition of probability,
\begin{equation}
  \label{eq:sum-prj-equals-1}
  \sum_{j=1}^m p(r_j) = 1,
\end{equation}
even when there are disconnected meanings whose probability should be zero.

From Equation \eqref{eq:prj-proportional-pirj} and \ref{eq:sum-prj-equals-1},
\begin{equation}
  \label{eq:definition-prj_second-model}
  p(r_j) = \frac{(1 - \delta_{\omega_j,0}) \pi(r_j)}{\rho}
\end{equation}
where
\begin{align}
  \label{eq:definition-rho}
  \rho &= \sum_{j=1}^m (1 - \delta_{\omega_j,0}) \pi(r_j) \\
       &= 1 - \sum_{j=1}^m \delta_{\omega_j,0} \pi(r_j). \nonumber
\end{align}

The conditional probability of choosing a word given a meaning is proportional to the number of meanings associated with that word or zero if that meaning is not associated to the word.

Applying
\begin{equation*}
  \sum_{i=1}^n p(s_i | r_j) = 1
\end{equation*}
to Equation \eqref{eq:prop-cond-prob_second-model} and recalling Equation \eqref{eq:definition-omegaphi} we obtain
\begin{equation}
  \label{eq:definition-cond-prob_second-model}
  p(s_i | r_j) = \frac{a_{i,j} \mu_i^\phi}{\omega_{\phi,j}}.
\end{equation}

The joint probability $p(s_i, r_j)$ is obtained by applying Equation \eqref{eq:definition-prj_second-model} and Equation \eqref{eq:definition-cond-prob_second-model} to the definition
\begin{equation*}
  p(s_i, r_j) = p(s_i | r_j) p(r_j),
\end{equation*}
obtaining
\begin{equation}
  \label{eq:definition-join-prob_second-model}
  p(s_i, r_j) = \frac{a_{i,j} (1 - \delta_{\omega_j,0}) \mu_i^\phi \pi(r_j)}{\rho \omega_{\phi,j}}.
\end{equation}

\subsubsection{Marginal word probability}

The probability of a word can be derived from Equation \eqref{eq:definition-join-prob_second-model} and applying
\begin{equation*}
  p(s_i) = \sum_{j=1}^m p(s_i, r_j),
\end{equation*}
obtaining
\begin{equation}
  \label{eq:definition-psi_second-model}
  p(s_i) = \frac{\mu_i^\phi \chi_i}{\rho}
\end{equation}
with
\begin{equation}
  \label{eq:definition-chi_second-model}
  \chi_i = \sum_{j=1}^m \frac{a_{i,j} (1 - \delta_{\omega_j,0}) \pi(r_j)}{\omega_{\phi,j}}.
\end{equation}

\subsubsection{Entropies}

Applying the definition of $H(R)$ (Equation \eqref{eq:definition-HR}) to $p(s_j)$ (Equation \eqref{eq:definition-prj_second-model}) we obtain
\begin{equation*}
  H(R) = - \sum_{j=1}^m \frac{(1 - \delta_{\omega_j,0}) \pi(r_j)}{\rho} \log \frac{(1 - \delta_{\omega_j,0}) \pi(r_j)}{\rho}.
\end{equation*}
Property \ref{th:definition-trick} can be applied immediately to simplify $H(R)$
\begin{equation*}
  H(R) = \log \rho - \frac{1}{\rho} \sum_{j=1}^m (1 - \delta_{\omega_j,0}) \pi(r_j) \log (1-\delta_{\omega_j,0}) \pi(r_j).
\end{equation*}
This is simplified further applying the convention $0 \log 0 = 0$
\begin{align}
  \label{eq:definition-HR_second-model}
  H(R) &= \log \rho - \frac{1}{\rho} \sum_{j=1}^m (1 - \delta_{w_j,0}) \pi(r_j) \log \pi(r_j) \\  
       &= \log \rho + \frac{1}{\rho} \left( H_\pi(R) + \sum_{j=1}^m \delta_{w_j,0} \pi(r_j) \log \pi(r_j) \right) \nonumber
\end{align}
where $H_\pi(R)$ is the entropy of the \emph{a priori} probabilities
\begin{equation*}
  %\label{eq:definition-HpiR_second-model}
  H_\pi(R) = -\sum_{j=1}^m \pi(r_j) \log \pi(r_j)
\end{equation*}

$H(S,R)$ can be derived by applying Equation \eqref{eq:definition-join-prob_second-model} to the information theory definition of $H(S,R)$ (Equation \eqref{eq:definition-HSR}).
Alternatively, it can also be derived by using the equality
\begin{equation*}
  H(S,R) = H(S|R) + H(R).
\end{equation*}
This reduces the problem to finding $H(S|R)$ using
\begin{equation*}
  H(S|R) = \sum_{j=1}^m H(S|r_j)p(r_j) 
\end{equation*}
and
\begin{equation*}
  H(S|r_j) = -\sum_{i=1}^n p(s_i|r_j) \log p(s_i|r_j).
\end{equation*}
Both approaches turn out to be quite cumbersome mathematically.
For the sake of brevity they are not shown here and can be consulted in Appendix \ref{sec:app_formulae_join-entropy_second-model}.
In either case, the resulting expression for $H(S,R)$ is
\begin{equation}
  \label{eq:definition-HSR_second-model}
  H(S,R) = \log \rho - \frac{1}{\rho} \sum_{j=1}^m (1 - \delta_{w_j,0}) \pi(r_j) \left[ \frac{\phi \nu_j}{\omega_{\phi,j}} + \log \frac{\pi(r_j)}{\omega_{\phi,j}} \right]
\end{equation}
with
\begin{equation}
  \label{eq:definition-nu}
  \nu_j = \sum_{i=1}^n a_{i,j} \mu_i^\phi \log(\mu_i).
\end{equation}

$H(S)$ is derived quite easily by applying Equation \eqref{eq:definition-psi_second-model} to the information theory definition (Equation \eqref{eq:definition-HS})
\begin{equation*}
  H(S) = -\sum_{i=1}^n \frac{\mu_i^\phi \chi_i}{\rho} \log \frac{\mu_i^\phi \chi_i}{\rho}
\end{equation*}

\begin{property}{}{sum-mui-chii-equals-rho}
  It is quite simple to see that
  \begin{equation*}
    \sum_{i=1}^n \mu_i^\phi \chi_i = \rho,
  \end{equation*}
  however this might not be immediately obvious as it has not appeared at any point until now.
  Appendix \ref{sec:app_formulae_proof-sum-HS_second-model} shows that this property indeeed holds.
\end{property}

Following Property \ref{th:sum-mui-chii-equals-rho}, Property \ref{th:definition-trick} can be applied here, obtaining
\begin{equation}
  \label{eq:definition-HS_second-model}
  H(S) = \log \rho - \frac{1}{\rho} \sum_{i=1}^n \mu_i^\phi \chi_i \log \mu_i^\phi \chi_i
\end{equation}

\subsubsection{Dynamic Equations}

The full derivation of the dynamic equations is given in Appendix \ref{sec:app_formulae_dynamic-equations_second-model}.
Here only the final expressions of the dynamic equations are given. Recall Section \ref{sec:model_math_first-model} for many useful definitions. In all dynamic equations, a mutation on $a_{i,j}$ is assumed.

Compact expressions of the entropies (Equations \eqref{eq:definition-HR_second-model}, \eqref{eq:definition-HSR_second-model} and \eqref{eq:definition-HS_second-model})
\begin{align}
  \label{eq:definition-HSR_second-model_compact}
  H(S,R) &= \log \rho - \frac{1}{\rho} X(S,R) \\
  \label{eq:definition-HS_second-model_compact}
  H(S) &= \log \rho - \frac{1}{\rho} X(S) \\
  \label{eq:definition-HR_second-model_compact}
  H(R) &= \log \rho - \frac{1}{\rho} X(R)
\end{align}
with
\begin{align}
  \label{eq:definition-XSR_second-model}
  X(S,R) &= \sum_{j=1}^m (1 - \delta_{\omega_j,0}) x(r_j) \\
  \label{eq:definition-XS_second-model}
  X(S) &= \sum_{i=1}^n x_{s_i} \\
  \label{eq:definition-XR_second-model}
  X(R) &= \sum_{j=1}^m (1-\delta_{\omega_j,0}) \pi(r_j) \log \pi(r_j) \\
  \label{eq:definition-xsi_second-model}
  x(s_i) &= \mu_i^\phi \chi_i \log \left( \mu_i^\phi \chi_i \right) \\
  \label{eq:definition-xrj_second-model}
  x(r_j) &= \pi(r_j) \left[ \frac{\phi \nu_j}{\omega_{\phi,j}} + \log \frac{\pi(r_j)}{\omega_{\phi,j}} \right].
\end{align}

Equations \eqref{eq:definition-aij_dynamic} ($a'_{i,j}$) and \eqref{eq:definition-mui_dynamic} ($\mu'_i$) remain the same as in the \firstmodel{}.
Equation \eqref{eq:definition-wj_dynamic} ($\omega'_j$) is unused in this model.

Equation \eqref{eq:definition-wphik_dynamic} ($\omega'_{\phi,l}$ for any $l$ such that $1 \leq l \leq m$) remains identical, while Equation \eqref{eq:definition-muphik_dynamic} ($\mu'_{\phi,k}$ for any $k$ such that $1 \leq k \leq n$) is unused in this model.

Recall as well the definition of the set of all edges to neighbors of $s_i$ and $r_j$ (Equation \eqref{eq:definition-Nij}). Two additional sets are defined for ease in the definition of these equations, $A_{i,j}(k)$ is the set of all meanings that are neighbors of $i$ ($\Gamma_S(i)$) (plus the meaning $r_j$ if not already included) that are also neighbors of $k$
\begin{equation}
  \label{eq:definition-A_second-model}
  A_{i,j}(k) = (\Gamma_S(i) \cup \set{r_j}) \cap \Gamma_S(k).
\end{equation}
The other set, $B_{i,j}(k)$ is the set of all words $s_k$ such that $A_{i,j}(k)$ is not the empty set.
The word $s_i$ is always included,
\begin{equation}
  \label{eq:definition-B_second-model}
  B_{i,j}(k) = \Set{s_k}{A_{i,j}(k) \neq \emptyset} \cup \set{s_i}
\end{equation}

The one mutation change equations for $\rho$, $\nu$ and $\chi$ are
\begin{equation}
  \label{eq:definition-rho_dynamic}
  \rho' = \rho - \delta_{\omega'_j,0} \pi(r_j) + \delta_{\omega_j,0} \pi(r_j),
\end{equation}
\begin{equation}
  \label{eq:definition-nu_dynamic}
  \nu'_l = \begin{cases}
    \nu_l + (1 - a_{ij}) {\mu'}_i^\phi \log \mu'_i - a_{ij}\mu_i^\phi \log \mu_i & \text{if}~l=j \\
    \nu_l - \mu_i^\phi \log \mu_i + {\mu'}_i^\phi \log \mu'_i & \text{if}~l\in \Gamma_S(i)~\text{and}~ l \neq j \\
    \nu_l & \text{otherwise}
  \end{cases}
\end{equation}
and
\begin{equation}
  \label{eq:definition-chi_dynamic}
  \chi'_k = \chi_k - \sum_{l \in A_{i,j}(k)} \frac{\pi(r_l)}{\omega_{\phi,l}} + \sum_{l \in A_{i,j}(k)} \frac{\pi(r_l)}{\omega'_{\phi,l}}.
\end{equation}
$\chi_i$ is always recalculated statically instead

These variables can be applied to Equations \eqref{eq:definition-xsi_second-model} and \eqref{eq:definition-xrj_second-model} directly to obtain their values.

The expressions of $X'(S)$, $X'(R)$ and $X'(S,R)$ used to dynamically update Equations \eqref{eq:definition-HR_second-model}, \eqref{eq:definition-HSR_second-model} and \eqref{eq:definition-HS_second-model} are
\begin{equation}
  \label{eq:definition-XR_second-model_dynamic}
  X'(R) = X(R) - \delta_{\omega'_j,0} \pi(r_j) \log \pi(r_j) + \delta_{\omega_j,0} \pi(r_j) \log \pi(r_j),
\end{equation}
\begin{equation}
  \label{eq:definition-XSR_second-model_dynamic}
  X'(S,R) = X(S,R) - \sum_{l \in \Gamma_S(i) \setminus \{ j \}} x(r_l) + \sum_{l \in \Gamma_S(i) \setminus \{ j \}} x'(r_l) - (1 - \delta_{\omega_j,0}) x(r_j) + (1 - \delta_{\omega'_j,0}) x'(r_j)
\end{equation}
and
\begin{equation}
  \label{eq:definition-XS_second-model_dynamic}
  X'(S) = X(S) - \sum_{o \in B_{i,j}(k)} x(r_o) + \sum_{o \in B_{i,j}(k)} x'(r_k)
\end{equation}

Again, the full derivation and logic behind these equations is given in Appendix \ref{sec:app_formulae_dynamic-equations_second-model}.
It is not included here for brevity, as it is a long not immediately obvious derivation.

\subsubsection{Extreme cases and invariants}
\label{sec:model_math_first-model_invariants}

For verification purposes (see Section \ref{sec:methods_verification}), the values of the entropies along with their invariants are also given here.

\paragraph{Extreme cases} These cases can be easily derived from Equations \eqref{eq:definition-HSR_second-model}, \eqref{eq:definition-HS_second-model} and \eqref{eq:definition-HR_second-model} ($H(S,R)$, $H(S)$ and $H(R)$ respectively) by assuming the corresponding condition and applying elementary algebra.

\begin{itemize}
\item For a single edge, $H(S), H(R), H(S,R) = 0$.
\item For a complete graph, $H(S) = \log n, H(R) = H_\pi(R), H(S,R) = H_\pi(R) + \log n$
\item For a one-to-one mapping of signals into meanings with $n=m$, $H(S), H(R), H(S,R) = H_\pi(R)$
\end{itemize}

\paragraph{Invariants} These invariants follow from basic information theory.
\cite{Cover1999}

\begin{itemize}
\item $0 \leq H(S) \leq \log n$
\item $0 \leq H(R) \leq H_\pi(R)$
\item $0 \leq H(S,R) \leq H_\pi(R) + \log n$
\end{itemize}

\subsection{Lower bound of the cost function}

Upper and lower bounds for the cost function $\Omega$ can be derived from the definition of Omega (Equation \eqref{eq:definition-Omega}) and the bound \cite{Cover1999}
\begin{equation*}
  0 \leq I(S,R) \leq \min(H(S),H(R)),
\end{equation*}
obtaining
\begin{align*}
  \Omega(\lambda) &\geq -\lambda \min(H(S),H(R)) + (1-\lambda) H(S) \\
                  &\geq (1-2\lambda)H(S)
\end{align*}
Knowing the bounds for $H(S)$ (Section \ref{sec:model_math_first-model_invariants}) we reach that, in general
\begin{equation}
  \label{eq:lower-bound-Omega}
  \Omega(\lambda) \geq
  \begin{cases}
    0                           & \text{if}~\lambda \leq 1/2 \\
    (1-2\lambda) \log \min(n,m) & \text{if}~\lambda \geq 1/2.
  \end{cases}
\end{equation}

\section{Computational aspect}
\label{sec:model_compute}

This section covers the computational side of the model.
This side is based on the mathematics covered in Section \ref{sec:model_math}.
Both models are seen separately, Section \ref{sec:model_compute_first-model} covers the \firstmodel{} while Section \ref{sec:model_compute_second-model} covers the \secondmodel{}.
Both sections cover the computational cost of computing the variables of the model, both completely (static calculation) and the change after a mutation to $a_{i,j}$ (dynamic calculation) as well as the changes that take place by treating the case $\phi=0$ separately.

A summary table comparing the computational costs of each of the models and the particular cases here is given at the end as Table \ref{tab:summary-computational}.

\subsection{The \firstmodel{}}
\label{sec:model_compute_first-model}

This section covers the computational cost of the calculation of the entropies of the \firstmodel{}.

\subsubsection{Static}

$A$, $E$, $\mu$ and $\omega$ are always calculated dynamically and updated whenever an edge is added or removed to the graph, as it is very simple to do so.
All entropies are calculated statically in a single loop iterating over every edge in $E$.
The joint entropy $H(S,R)$ (Equation \eqref{eq:definition-HSR_first-model}) and the normalization factor $M_\phi$ (Equation \eqref{eq:definition-Mphi}) are updated on every step of the loop.
$\mu_\phi$ and $\omega_\phi$ are updated on every iteration as well.
$H(S)$ (Equation \eqref{eq:definition-HS_first-model}) is only updated when $\mu_{\phi,i}$ for a particular word $s_i$ has been fully recalculated.
$H(R)$ (Equation \eqref{eq:definition-HR_first-model}) is updated in the same way as $H(S)$, whenever $\omega_{\phi,j}$ for a particular meaning $r_j$ has been fully recalculated.
The cost of the static calculation of entropies is $\bigO{M}$.

\subsubsection{Dynamic}

In the case of the dynamic calculation of entropies, the algorithmic cost is dominated by the computation of $X'(S)$, $X'(R)$ and $X'(S,R)$ (Equations \eqref{eq:definition-XSR_first-model_dynamic}, \eqref{eq:definition-XS_first-model_dynamic} and \eqref{eq:definition-XR_first-model_dynamic}).
It can be seen immediately that looping over all the neighbors of $s_i$ and $r_j$ is necessary in order to update all entropies.
The cost of the dynamic calculation of entropies is then $\bigO{\max(\mu_i,\omega_j)}$.

\subsubsection{The case $\phi=0$}

In order to speed up calculations, simpler equations for the case $\phi=0$ are derived.
$X'(S,R)$ (Equation \eqref{eq:definition-XSR_first-model_dynamic}) is no longer used, as $H(S,R) = \log M$ when $\phi=0$ (see Equation \eqref{eq:definition-HSR_first-model}).

$X'(S)$ (Equation \eqref{eq:definition-XS_first-model_dynamic}) becomes
\begin{equation}
  \label{eq:definition-XS_first-model_dynamic-phi0}
  X'(S) = X(S) - \mu_i \log \mu_i + \mu'_i \log \mu'_i
\end{equation}
using the convention $0 \log 0 = 0$ where necessary.

Similarly to $X'(S)$, $X'(R)$ (Equation \eqref{eq:definition-XR_first-model_dynamic}) becomes
\begin{equation}
  \label{eq:definition-XR_first-model_dynamic-phi0}
  X'(R) = X(R) - \omega_j \log \omega_j + \omega'_j \log \omega'_j
\end{equation}

In the case $\phi=0$, $M_\phi$ (Equation \eqref{eq:definition-Mphi}) becomes $M$, the number of edges in the graph.

As can be seen from equations \eqref{eq:definition-XS_first-model_dynamic-phi0} and \eqref{eq:definition-XR_first-model_dynamic-phi0}, the cost of the dynamic calculation is greatly reduced when $\phi=0$, becoming $\bigO{1}$.

\subsection{The \secondmodel{}}
\label{sec:model_compute_second-model}

This section overs the computational cost of the calculation of the entropies of the \secondmodel{}.

\subsubsection{Static}

As with the \firstmodel{} (see Section \ref{sec:model_compute_first-model}), $A$, $E$, $\mu$ and $\omega$ are dynamically updated whenever an edge is added or removed from the graph while all other variables, including entropies, are calculated statically.

As can be seen from Equations \eqref{eq:definition-HR_second-model}, \eqref{eq:definition-HSR_second-model} and \eqref{eq:definition-HS_second-model}, a loop over every edge in the graph is needed in order to recalculate all entropies and variables.
Unlike the \firstmodel{}, however, this loop needs to be repeated twice.

During the first run, $\omega_\phi$ (Equation \eqref{eq:definition-omegaphi}) and $\nu$ (Equation \eqref{eq:definition-nu}) are updated on every iteration.
$\rho$ (Equation \eqref{eq:definition-rho}), $H(R)$ (Equation \eqref{eq:definition-HR_second-model}) and $H(S,R)$ (Equation \eqref{eq:definition-HSR_second-model}) are calculated only on the iterations where $\omega_{\phi,j}$ and $\nu_j$ have been completely calculated for a single meaning $r_j$.

This leaves $\chi$ and $H(S)$ to be calculated during the second run of the loop over all edges.
The computation of $\chi_i$ (Equation \eqref{eq:definition-chi_second-model}) for a word $s_i$ requires $\omega_{\phi,j}$ to be fully computed for every meaning $r_j$.
This is the reason for running two separate loops, fully calculating $\omega_\phi$ in one run so that $\chi$ may be calculated in the other.
$H(S)$ (Equation \eqref{eq:definition-HS_second-model}) is updated only on the iterations where $\chi_i$ has been completely calculated for a single word $s_i$.

The cost of the static calculation of entropies is the same as in the \firstmodel, $\bigO{M}$.

\subsubsection{Dynamic}

The dynamic calculation of entropies is dominated by the computation of $X'(S)$, $X'(S,R)$ and $\chi$ (Equations \eqref{eq:definition-XSR_second-model}, \eqref{eq:definition-XS_second-model} and \eqref{eq:definition-chi_dynamic}).
These three values iterate on three different sets: $\Gamma_S(i)$, $A_{i,j}$ and $B_{i,j}$ (Equations \eqref{eq:definition-Gamma-S}, \eqref{eq:definition-A_second-model} and \eqref{eq:definition-B_second-model}).
It is clear from the definition of set $A_{i,j}$ is lesser or similar (one more element) in size to the set $\Gamma_S(i)$, but $\Gamma_S(i)$ will usually be bigger.
The size of the set $B_{i,j}$ depends on the structure of the graph.

The cost of the dynamic calculation of entropies is then $\bigO{\max(\mu_i,|B_{i,j}|)}$

\subsubsection{The case $\phi=0$}

In order to speed up calculations, simpler equations for the case $\phi=0$ are derived. Much of the dependency on neighboring nodes is removed.

$X(R)$ and $\rho$ (Equations \eqref{eq:definition-XR_second-model_dynamic} and \eqref{eq:definition-rho_dynamic}) are not affected as they do not depend on $\phi$.
They remain simple.

$x(r_j)$ becomes
\begin{equation}
  \label{eq:definition-xrj_second-model_dynamic-phi0}
  (1-\delta_{w_j,0}) \pi(r_j)\log\frac{\pi(r_j)}{\omega_j}
\end{equation}
and so no longer depends on neighbors of $r_j$.
Consequently, $X(S,R)$ also no longer depends on a set of neighbors and becomes
\begin{equation}
  \label{eq:definition-XSR_second-model_dynamic-phi0}
  X'(S,R) = X(S,R) - x(r_j) + (1-\delta_{\omega'_j,0}) x'(r_j).
\end{equation}

$\chi_i$ (Equation \eqref{eq:definition-chi_second-model}) is simplified but still depends on the neighborhood of $s_i$ in order to be updated
\begin{equation}
  \label{eq:definition-chi_second-model_dynamic-phi0}
  \chi'_k = \begin{cases}
    \chi_i + (1-a_{i,j}) \frac{\pi(r_j)}{\omega'_j} - a_{i,j} \frac{\pi(r_j)}{\omega_j} & \text{if}~k=1 \\
    \chi_k - \frac{\pi(r_j)}{\omega_j} + \frac{\pi(r_j)}{\omega'_j} & \text{if}~k\in\Gamma_R(j)~\text{and}~k\neq i \\
    \chi_k & \text{otherwise}.
  \end{cases}
\end{equation}
As $X(S)$ depends on $\chi$ (Equation \eqref{eq:definition-XS_second-model}), $X'(S)$ becomes
\begin{equation}
  \label{eq:definition-XS_second-model_dynamic-phi0}
  X'(S) = X(S) - \sum_{k \in \Gamma_R(j) \cup \set{i}} x(s_i) + \sum_{k \in \Gamma_R(j) \cup \set{i}} x'(s_i)
\end{equation}
with (see Equation \eqref{eq:definition-xsi_second-model})
\begin{equation}
  \label{eq:definition-xsi_second-model_dynamic-phi0}
  x(s_i) = \chi_i \log \chi_i.
\end{equation}

$X(S)$ and $\chi$ need to iterate the neighbors of $s_i$, and so the computational complexity is $\bigO{\mu_i}$.

%%% Local Variables:
%%% mode: latex
%%% TeX-master: "tfm"
%%% End:
