\chapter{Code}
\label{cha:code}

\section{repo}
\begin{itemize}
\item afegir repositori amb el codi
\end{itemize}

\section{Program parameters}
\label{sec:app_code_program-parameters}

Here all the parameters that can be specified to the tool created as part of this thesis are explained.

\begin{itemize}
\item
  $n$ and $m$, the size of the two sets in the bipartite graph
\item
  $\phi$ the parameter that was added to generalize the older models
\item
  Whether or not to use constant \emph{a priori} probabilities for the meanings.
  Using them implies using the \secondmodel{}, not using them means using the \firstmodel{}.
\item
  Whether unlinked objects are allowed or not.
\item
  The $\pi$ parameter, the distribution of the \emph{a priori} probabilities, can be specified in a number of ways if these probabilities are indeed being used.
  In addition to those specified in Section \ref{sec:methods_model-implementation}, it is allowed to manually specify an arbitrary probability distribution.
\item
  Which $\lambda$ parameters to test, the program will generate data for all the specified $\lambda$ paramters.
 They can be specified either as a range or manually one by one.
\item
  The number of realizations to be done.
  For each value of $\lambda$, that many optimization processes will be carried out and the results averaged.
\item
  The initial graph that the optimization starts from.
  This can be a \randgraph{n}{m}{p} or \randgraph{n}{m}{e} random bipartite graph (see Section \ref{sec:methods_model-implementation}), a complete graph or a one to one bijection.
  In that last case, only the first $\min(n,m)$ words and meanings are connected.
\item
  Whether to use the static or dynamic equations.
\item
  The number of mutations.
  Either a constant number of or random binomial mutations with a given probability.
\item
  The stop condition.
  Either of the weak or strong conditions, as well as the option of manually specifying a number.
  It is always the number of failures to improve $\Omega$.
\item
  A random seed can be manually specified or left random (even if random, the seed used is always logged).
  Whether to emit or not certain results.
  By default the program emits several csv files corresponding to the information theoretical plots as a function of lambda and the statistics for every value of lambda computed.
  It also outputs every generated graph in every realization.
  Any other measure could be obtained from these graphs.
\end{itemize}

%%% Local Variables:
%%% mode: latex
%%% TeX-master: "tfm"
%%% End:
