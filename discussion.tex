\chapter{Discussion}
\label{cha:discussion}
Aquesta part és important, ha de quedar clara. Parlar tambe sobre limitacions: quins resultats son contundents i quins febles.

Analisi critic, interpretar, lligar amb context de coneixements: relacio paraula edat frequencia i altres models de zipf com es relacion amb ells, com ho expliquen. Distincio entre model que explica i model que descriu. No hi ha explicació sense teoria. Teoria es una serie de principis. No tots els models son explicacions pk no poden fer prediccions. E.g., random typing no diu res de significats no pot explicar aprenentatge en nens.
\section{mates (linguistica quantitativa)}
\subsection{zipf law en regressio lineal del log-log?}
\begin{itemize}
\item mirar grafiques dels dos models, variants
\end{itemize}
\subsection{relacio edat-frequencia}
\begin{itemize}
\item grafiques mostren que paraules mes antigues son mes frequents
\item correspon amb zipf (llibre)
\end{itemize}
\section{computacional}
\begin{itemize}
\item lligar amb biosemiotics
\end{itemize}
\subsection{minim local}
\begin{itemize}
\item el llei zipf minim local? stop condition
\item fins a quin punt recuperar lleis depen de parametres?
\end{itemize}
\section{future work}
\subsection{altres metodes d'optimitzacio}
\begin{itemize}
\item simulated annealing (non zero temperature)
\item gradient descent (make $\Omega$ derivable)
\end{itemize}
\subsection{altres valors de phi}
hem fet 1 i 0 per altres, 0.5, 1.5...
\subsection{vocabulary nens segon model}
biosemiotics ha fet primer model nomes
\subsection{evitar error numeric encara mes}
\begin{itemize}
\item subcalculs del metode dinamic que es podrien fer estatics
\item ja es fa un valor de $\chi$
\end{itemize}

%%% Local Variables:
%%% mode: latex
%%% TeX-master: "tfm"
%%% End:
