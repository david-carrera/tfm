\begin{table}
  \centering
  \begin{adjustbox}{max width=\textwidth}
    \begin{tabular}{lcc}
      \toprule
                             & \Firstmodel{} & \Secondmodel{} \\
      \midrule
      Static                 & $\bigO{M}$                    & $\bigO{M}$                   \\
      Dynamic ($\phi\neq 0$) & $\bigO{\max(\mu_i,\omega_j)}$ & $\bigO{\max(\mu_i,|B_{i,j}|)}$ \\
      Dynamic ($\phi=0$)     & $\bigO{1}$                   & $\bigO{\mu_i}$                \\
      \bottomrule
    \end{tabular}
  \end{adjustbox}
    \caption{
      Summary table of the computational cost of each model and for particular cases.
      The columns indicate which of the two models and the rows the specific case for that model (static, dynamic general case for any value of $\phi$ and dynamic for the specific case of $\phi=0$ where simplifications are possible).
      The \firstmodel{} with $\phi=0$ originates from \cite{Ferrer2005a}.
      The \secondmodel{} with $\phi=0$ from \cite{Ferrer2003a}.
      The generalization where $\phi \neq 0$ for the family of models was introduced in \cite{Ferrer2018a} but without any simulation results.
      Each cell shows the computational complexity of updating the value of the cost function after mutating $a_{ij}$ (that is, the word $s_i$ and the meaning $r_j$ become connected from being disconnected or become disconnected from being connected).
      $M$ is the number of connections in the model.
      $\mu_i$ is the number of neighbors of the word ($s_i$) and $\omega_j$ the number of neighbors of the meaning ($r_j$).
      $|B_{i,j}|$ is the number of words that have at least one neighbor in common with $s_i$ including $r_j$.
    }
  \label{tab:summary-computational}
\end{table}

%%% Local Variables:
%%% mode: latex
%%% TeX-master: "../tfm"
%%% End:
