\begin{table}
  \centering
  \begin{adjustbox}{max width=\textwidth}
    \begin{tabular}{llSS[scientific-notation=true]}
      \toprule
      plot & law & $a$ & $k$ \\ 
      \midrule
      a & word frequency ($\alpha$) & 2.6176507 & 0.6341345 \\ 
      %b & \redtxt{word frequency (cumulative)} & 0.3812498 & 0.0021506 \\ 
      b & meaning distribution ($\gamma$) & 1.1623482 & 37.4616016 \\ 
      c & meaning frequency ($\delta$) & -0.5600378 & 158.5629462 \\ 
      %f & \redtxt{meaning distribution} & 1.1623482 & 37.4616016 \\ 
      %g & \redtxt{meaning distribution (cumulative)} & 0.3954554 & 0.0250000 \\ 
      \bottomrule
    \end{tabular}
  \end{adjustbox}    
  \caption{Table showing the exponent and factor of the power laws fitted for the \secondmodel{} with $\phi=0$ and a random bipartite graph as the initial condition (Figure \ref{fig:fitting_insideLambda_uniform_phi0_nm400_dynamic_randomBipartite_allowUnlinked})
    In this table, $\alpha \approx 2.6$, $\delta \approx 0.56$ and $\gamma \approx 1.2$.
    The relationship between these values (Equation \eqref{eq:relation-exponents}) holds approximately.
    The exponents for $\alpha$ and $\gamma$ are not exactly the ones expected but the exponent for $\delta$ is very close.
  } 
  \label{tab:fitting_insideLambda_uniform_phi0_nm400_dynamic_randomBipartite_allowUnlinked}
\end{table}

%%% Local Variables:
%%% mode: latex
%%% TeX-master: "../tfm"
%%% End:
